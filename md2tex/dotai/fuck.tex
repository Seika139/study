

\newpage
\section*{指定課題}

\subsection*{1}
\[\rm \frac{CL_H}{F_H} = \frac{Dose}{AUC} = f_B \cdot CL_{int,H}\]
より、$\rm CL_{int,H}$は$\rm AUC$に反比例する。したがって、肝固有クリアランスは
\[\frac{1}{1.6} = 62.5 \%\]
に低下する。

\subsection*{2}
\[\rm \%CL = \frac{1}{1+\frac{10 \times 0.1}{0.25}} = 20 \%\]

\subsection*{3}
OATP1B3の寄与率$x$は以下のように求められる。
\[0.2x + (1-x) = 0.625\]
\[x = 46.875\%\]

\subsection*{4}
OATP1B3の寄与率$y$は以下のように求められる。
\[0.2 \times 0.46875 + 0.2y + (1-0.46875-y) = \frac{1}{2.7}\]
\[y = 105.46\%\]

\subsection*{5}
静脈内投与だった場合は、
\[\rm AUC = \frac{Dose}{CL_H} = Dose \cdot \frac{Q_H + f_B \cdot CL_{int,H}}{Q_H \cdot f_B \cdot CL_{int,H}}\]
となる。肝固有クリアランス律速の場合は
\[\rm AUC = \frac{Dose}{f_B \cdot CL_{int,H}} \]
となるので経口投与の場合と変わらないと考えらるが、血流律速の場合は
\[\rm AUC =\frac{Dose}{Q_H} となるので、AUCが変化しないくなると考えられる。

\subsection*{6}
\begin{itemize}
\item 代謝酵素Xの寄与が他の代謝酵素に比べて小さい。
\item 薬物Aには再取り込みも大きくなる。
\end{itemize}
などが考えられる。

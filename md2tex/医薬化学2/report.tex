\documentclass[a4paper,papersize,dvipdfmx]{jsarticle}
\usepackage{ascmac}
\usepackage{mathtools, amssymb,bm}
\usepackage{comment}
\usepackage[hiresbb]{graphicx}
\usepackage{tcolorbox,color}
\usepackage{here}
\tcbuselibrary{raster,skins,breakable}

\newcommand{\pic}[1]{\begin{center} \includegraphics[width=1.0\linewidth,clip]{#1} \end{center}}   %写真用
\newcommand{\pict}[2]{\begin{center} \includegraphics[width= {#2} cm]{#1} \end{center}}   %写真用
\newcommand{\piccap}[3]{\begin{figure}[H] \centering \includegraphics[width= {#2} cm]{#1} \caption{#3} \label{fig {#1}} \end{figure}} %キャプションつき画像
\newcommand{\redunderline}[1]{\textcolor{red}{\underline{¥textcolor{black}{#1}}}}   %赤いアンダーライン
\newcommand{\mon}[1]{\item[({#1})] \ }
\newcommand{\ctext}[1]{\raise0.2ex\hbox{\textcircled{\scriptsize{#1}}}}%文字を丸囲みする(2桁の数字までならいける)

% 画像を貼る時はjpgかjpegで、pngはうまくいかない時もある

%\itemを四角で囲った数字にする場合は以下のコメントアウトを消す
%\renewcommand{\labelenumi}{\textbf{\framebox[1.5zw]{\theenumi}}}


%enumerateの2階層めのカウンタを1,2,3, にする時は以下のコメントアウトを消す
\renewcommand{\theenumii}{\arabic{enumii}}

%enumerateのカウンタについては以下を参照
% http://www3.otani.ac.jp/fkdsemi/pLaTeX_manual/kajyo.html


%enumerateの番号の出力形式を変更するには、カウンタの値を出力する命令を定義し直す。
%レベル	カウンタ	出力する命令	デフォルトの出力
%1	enumi	¥theenumi	アラビア数字(1,2,3,・・・)
%2	enumii	¥theenumii	小文字のアルファベット(a,b,c,・・・)
%3	enumiii	¥theenumiii	小文字のローマ数字(小文字のローマ数字(ⅰ,ⅱ,ⅲ,・・・)
%4	enumiv	¥theenumiv	大文字のアルファベット(A,B,C,・・・)
%例:¥enumiカウンタを大文字のローマ数字で出力する設定
% ¥renewcommand{¥theenumi}{¥Roman{enumi}}

% 番号の出力形式
%命令	出力形式
%¥arabic	アラビア数字(1、2、3、・・・)
%¥roman	ローマ数字(ⅰ、ⅱ、ⅲ、・・・)
%¥Roman	ローマ数字(Ⅰ、Ⅱ、Ⅲ、・・・)
%¥alph	アルファベット(a、b、c、・・・)
%¥Alph	アルファベット(A、B、C、・・・)




\begin{document}

\title{医薬化学2 安達先生のレポート課題}
\author{10191043 鈴木健一}
%作成日を入れる場合は消す
\date{}
\maketitle

%以下の3つからフォントサイズを選択するとよい
%\footnotesize
%\small
%\normalsize

\section*{1. FTY720がブロックバスターになった理由}
FTY720がブロックバスターとして成功を収めたのには複数の幸運が重なったからと言える。
そもそもISP-1を発見するにはシクロスポリンの産生菌が誤って分類命名されたこと、
さらには企業間での人の繋がりがあったという偶然がある。

その後、ISP-1を出発物質として様々な半合成品を合成し、ISP-1-29やISP-1-36などが合成された。
この時に初期に合成されたISP-1からISP-1-36に進むにつれて作用メカニズムがSPT阻害からSPI受動体作動へと知らぬうちに変化したが、
結果的に同じ活性を得られたのが最も大きな幸運と言えるだろう。
もし先にSPT阻害のメカニズムが先にわかっていればTDDの手法により
SPT阻害を目的として創薬が行われていたはずなので、
FTY720が発見されることはなかったと言えるだろう。

\pict{sl1.JPG}{8}

\section*{2. 自分が薬を作るならPDDとTDDのどちらを選択するか。}
TDDはゴールがわかった上で目的となる薬を作るアプローチなのでより低コストで成果を得られると思われたが、
実際にはTDDで大きな成果を収められていない。昔ながらのPDDで目的物質を探した方がダイナミックな発見ができるかもしれないので、
自分はPDDで創薬研究をしたいと思う。

\pict{sl2.JPG}{8}

\section*{3. PDDとTDDの長所と短所}
PDDは昔ながらの創薬手法で、標的タンパク質を設定しないで、より病態に近い細胞や動物を使って、
それらの変化や反応を見ることにより薬物を創製する方法である。

TDDは予め標的タンパク質を設定した上で薬物の探索を行う手法。
1990年代に始まったHTS技術やコンビナトリアル化学、ゲノム創薬といった最先端技術の発展の上に進歩を遂げた手法で
最近まで世界中の創薬会社で採用されてきた創薬の王道であった。
ところが鳴り物入りで登場したこの手法はそれほど大きな成果を収められなかった。

\pict{sl3.JPG}{8}

\section*{4. ファイストインクラス薬剤を創るにはPDDとTDDのどちらが良いか。}
PDDとTDDのどちらもメリットがあり、一概にどちらがいいとは言い切れないが、
標的タンパク質を設定しないPDDであればメガファーマ以外の企業にもチャンスがあるので、
PDDによる創薬でファイストインクラスを創生するのが良いと思う。

\end{document}

\documentclass[a4paper,papersize,dvipdfmx]{jsarticle}
\usepackage{ascmac}
\usepackage{mathtools, amssymb,bm}
\usepackage{comment}
\usepackage[hiresbb]{graphicx}
\usepackage{tcolorbox,color}
\tcbuselibrary{raster,skins,breakable}

\newcommand{\pic}[1]{\begin{center} \includegraphics[width=1.0\linewidth,clip]{#1} \end{center}}   %写真用
\newcommand{\pict}[2]{\begin{center} \includegraphics[width= {#2} cm]{#1} \end{center}}   %写真用
\newcommand{\redunderline}[1]{\textcolor{red}{\underline{¥textcolor{black}{#1}}}}   %赤いアンダーライン
\newcommand{\mon}[1]{\item[({#1})] \ }
\newcommand{\ctext}[1]{\raise0.2ex\hbox{\textcircled{\scriptsize{#1}}}}%文字を丸囲みする(2桁の数字までならいける)

%\itemを四角で囲った数字にする場合は以下のコメントアウトを消す
%\renewcommand{\labelenumi}{\textbf{\framebox[1.5zw]{\theenumi}}}


%enumerateの2階層めのカウンタを1,2,3, にする時は以下のコメントアウトを消す
\renewcommand{\theenumii}{\arabic{enumii}}

%enumerateのカウンタについては以下を参照
% http://www3.otani.ac.jp/fkdsemi/pLaTeX_manual/kajyo.html


%enumerateの番号の出力形式を変更するには、カウンタの値を出力する命令を定義し直す。
%レベル	カウンタ	出力する命令	デフォルトの出力
%1	enumi	¥theenumi	アラビア数字(1,2,3,・・・)
%2	enumii	¥theenumii	小文字のアルファベット(a,b,c,・・・)
%3	enumiii	¥theenumiii	小文字のローマ数字(小文字のローマ数字(\UTF{2170},\UTF{2171},\UTF{2172},・・・)
%4	enumiv	¥theenumiv	大文字のアルファベット(A,B,C,・・・)
%例:¥enumiカウンタを大文字のローマ数字で出力する設定
% ¥renewcommand{¥theenumi}{¥Roman{enumi}}

% 番号の出力形式
%命令	出力形式
%¥arabic	アラビア数字(1、2、3、・・・)
%¥roman	ローマ数字(\UTF{2170}、\UTF{2171}、\UTF{2172}、・・・)
%¥Roman	ローマ数字(\UTF{2160}、\UTF{2161}、\UTF{2162}、・・・)
%¥alph	アルファベット(a、b、c、・・・)
%¥Alph	アルファベット(A、B、C、・・・)




\begin{document}

\title{薬化学教室 実習レポート}
\author{10191043 鈴木健一}
%作成日を入れる場合は消す
\date{}
\maketitle

%以下の3つからフォントサイズを選択するとよい
%\footnotesize
\small
%\normalsize


\section*{1. ニトロベンゼンのニトロ化 (2019/4/22)}

\subsection*{目的}
芳香族求電子置換反応であるニトロ化を行う。メタ体が主生成物であるが、オルト体やパラ体も副生することを確認する。ニトロベンゼンのニトロ化は通常のニトロ化の条件(濃硫酸と硝酸)より強い条件が必要である。

\subsection*{実験方法}

\begin{enumerate}
\item ドラフトで濃硫酸 7 mL (12.5 g)を 100 mLナスフラスコに秤りとる。

\item ドラフトで氷冷しながらで発煙硝酸 5 mL (7.5 g)を駒込ピペットで 0.5 mL ずつ加える。

\item 実験台に運び、簡易排気装置の下で氷冷する。

\item ニトロベンゼン 5.0 g を氷水で冷却しながら駒込ピペットで容器をふり混ぜながら 1 mL ずつ加える。

\item 反応液を図のような fume chamber 下で換気しながら、ときどき手で容器をふり混ぜながら湯浴(80~90 °C)中で30分加熱する。このとき、湯気の水滴が反応容器に入らないように気をつける。

\item 反応容器を氷水につけて室温まで冷やし、200~300 mL の氷が浮いた水が入った 500 mL ビーカーにゆっくりと注ぎ、析出した結晶をスパーテルで砕く。なすフラスコは水で洗浄する。

\item グラスフィルターと吸引装置を用いて結晶を濾取し、水で結晶についた酸性の溶媒を洗いながし、スパーテルで結晶を細かく砕く。

\item 結晶を濾紙上に移し、別の濾紙に挟んで水を吸い取る。

\item 粗結晶を10分間減圧乾燥し、重さ、色、形状を記録する。

\item エタノール(15~30 mL)を用いて再結晶を行う。一番晶は氷冷して桐山ろうとで濾取し、冷却したEtOHで洗う。二番晶も同様に濾取して一番晶とともに10分間減圧乾燥する。その後重さなどを記録する。

\item 母液、 一番晶、二番晶と他のサンプルを用いてTLCを行う。

\end{enumerate}
\pict{imgs/1-zu.jpeg}{5}
\subsection*{結果}
\begin{itemize}
\item 今回用いたニトロベンゼンは 5.04 g だった。
\item 手順5において湯浴の中で加熱を始めると反応溶液の色が濃くなった。
\item その後加熱を続けると次第に黄色い沈殿が確認されるようになった。
\item 手順6において溶液を氷浴すると溶液の黄色が薄くなった。

\end{itemize}
\subsubsection*{粗結晶}
\begin{itemize}
\item 重さ : 7.08 g
\item 色 : 薄い黄色
\item 形状 : 粉末状
\item 収率 : $(7.08 \times 123.11) / (5.04 \times 168.11) = 1.029  \Longrightarrow 103 \%$

\end{itemize}
\subsubsection*{一番晶}
\begin{itemize}
\item 重さ : 5.09 g
\item 色 : 薄い黄色
\item 形状 : 粉末状
\item 融点 :  $77\sim81 ℃,  78\sim79.5 ℃$
\item 収率 : $(5.09 \times 123.11) / (5.04 \times 168.11) = 0.740  \Longrightarrow 74 \%$

\end{itemize}
\subsubsection*{二番晶}
\begin{itemize}
\item 重さ : 0.17 g
\item 色 : 薄い黄色
\item 形状 : 粉末状
\item 融点 : 57~58 ℃
\item 収率 : $(0.17 \times 123.11) / (5.04 \times 168.11) = 0.02470  \Longrightarrow 2.5 \%$

\end{itemize}
\subsubsection*{TLC}
\begin{itemize}
\item 展開溶媒 ... 酢酸エチル : $n-$hexane = 1:4

左から「一番晶、二番晶、$o-$ジニトロベンゼン、$m-$ジニトロベンゼン、$p-$ジニトロベンゼン、ニトロベンゼン、母液1、母液2」の順でスポットした。

\pict{imgs/1-1.jpeg}{6}

\begin{table}[h]
\begin{tabular}{|c|c|c|c|c|c|c|c|c|}
\hline
  & 一番晶  & 二番晶  & オルト体 & メタ体  & パラ体  & ニトロベンゼン & 母液1  & 母液2  \\ \hline
上 & 0.44 & 0.44 &      &      & 0.44 & 0.44    & 0.44 & 0.44 \\ \hline
中 & 0.28 & 0.28 &      & 0.28 &      &         & 0.28 & 0.28 \\ \hline
下 & 0.12 & 0.12 & 0.12 &      &      &         & 0.12 & 0.12 \\ \hline
\end{tabular}
\end{table}

\end{itemize}
\subsection*{考察}
TLCの結果から上段のスポットは$p-$ジニトロベンゼン、またはニトロベンゼンのみ呈色することがわかり、中段スポットは$m-$ジニトロベンゼンのみ、下段のスポットは$o-$ジニトロベンゼンに由来することがわかる。再結晶をしたところ一番晶も二番晶も上段と下段が薄く呈色したことから再結晶して得られた個体は純粋な$m-$ジニトロベンゼン出ないことがわかる。このように、オルト・メタ・パラでジニトロベンゼンのRf値が異なるのは双極子モーメントの違いに由来する。

\pict{imgs/1-1.jpeg}{6}

ニトロ基は電子求引性であるため、図のように$o-$ジニトロベンゼンが一番双極子モーメントが強く、$p-$ジニトロベンゼンが最も双極子モーメントが弱くなる。したがって極性が最も強い$o-$ジニトロベンゼンが最もRf値が小さくなり、$p-$ジニトロベンゼンが最もRf値が大きくなる。

今回の実験で皮膚に硝酸がかかると黄色く変色する。これは以下で示すキサントプロテイン反応が起こっているからである。皮膚のチロシン、フェニルアラニン、トリプトファンといったベンゼン環を持つ芳香族アミノ酸がニトロ化することで呈色反応が起きている。

\pict{imgs/1-nitro.png}{7}


\subsection*{課題}
\subsubsection*{(1) - 1 反応機構を説明する}
\pict{imgs/1-hk.jpeg}{10}

\subsubsection*{(1) - 2 メタ体が主生成物となる理由}
\pict{imgs/1-k2.jpeg}{10}
オルト体・メタ体・パラ体は以下のように共鳴構造をとるが、オルト体とパラ体では一番右の共鳴構造の赤い部分で正電荷が反発するため不安定となる。したがって不安定な状態ができないメタ体が最も安定となり主生成物となる。


\section*{2. Reimer-Tiemann反応 (2019/4/23)}
\subsection*{目的}
クロロホルムから塩基によって生成するジクロロカルベンを求電子試薬とする芳香族求電子置換反応に分類される人名反応を行う。反応生成物の混合物をカラムクロマトグラフィーで分離し、さらに油性物質を結晶性誘導体に用いて同定する。

\subsection*{実験方法}

\begin{enumerate}
\item 50 mL ナスフラスコにフェノール(2 g)を秤りとり、室温で撹拌しながら、12 g の水酸化ナトリウムを 12 mL の水に溶かしたものを加える。

\item 図のように還流装置を組み立て、反応溶液を80~90℃の油浴で加熱しながら、ジムロートの上の口からクロロホルム 4 g を一気に加える。反応が激しく吹き上げるときは油浴から反応容器を出す。10分後にクロロホルムを 2 g 加え、さらに15分後に 2 g 加えて30分還流する。

\item 反応液を室温まで冷やし、さらに氷水で氷冷する。

\item 25\% v/v 希硫酸(20~30 mL)を加え、pH=1の強酸性にする。このとき、無機塩が析出する。中和熱を完全に冷却し、フラスコ内の固体を砕く。

\item 反応溶液のすべてを分液ロートに移し、さらに水(20 mL)で洗いこむ。50 mL のAcOEtで3回抽出する。AcOEt層を40 mL の水で2回洗う。
\item $\rm Na_2SO_4$で脱水し、濾過したのちにTLCで生成物を調べる。生成物の票品と比較して出発物質のフェノールや副生成物を確認する。
\item 溶媒を濃縮し二連球スプレーで30回ほど溶媒を吹き飛ばして、粗生成物の重さを記録する。濃縮の際、途中で30 mLのナスフラスコに移し、AcOEtで洗いこみ、2~3 gになるまで濃縮する。

\end{enumerate}
\pict{imgs/2-zu.jpeg}{4}
\subsection*{結果}

\subsubsection*{収率}
\[(2.68 \times 94.11) \div (2.05 \times 122.11) = 1.01 \Longrightarrow 101 \%\]

\subsubsection*{TLC}
*表の数値はRf値を表す

\begin{table}[h]
\begin{tabular}{|c|c|c|c|}
\hline
酢酸エチル & 上    & 中    & 下    \\ \hline
1度打ち  & 0.84 & 0.61 & 0.32 \\ \hline
2度打ち  & 0.84 & 0.61 & 0.32 \\ \hline
オルト体  & 0.84 & -    & -    \\ \hline
パラ体   & -    & -    & 0.32 \\ \hline
フェノール & -    & 0.61 & -    \\ \hline
\end{tabular}
\end{table}

\begin{table}[h]
\begin{tabular}{|c|c|c|c|}
\hline
ヘキサン  & 上    & 中    & 下    \\ \hline
1度打ち  & 0.59 & 0.18 & -    \\ \hline
2度打ち  & 0.59 & 0.18 & 0.03 \\ \hline
オルト体  & 0.56 & -    & -    \\ \hline
パラ体   & -    & -    & 0.03 \\ \hline
フェノール & -    & 0.15 & -    \\ \hline
\end{tabular}
\end{table}

\subsection*{考察}
今回の実験ではまずフェノールに水酸化ナトリウムを加えて塩基性条件にする。この際塩としてナトリウムフェノキシドが生成する。また塩基性条件のまま反応を進めていると以下に示す反応機構にあるようにサリチルアルデヒドが塩の状態で析出してしまうので、実験では希硫酸を加えている。この時無機塩である$\rm Na_2SO_4$が生じる。

\subsection*{課題}
\subsubsection*{(2)-1 反応機構を説明する}

\pict{imgs/2-hk1.jpeg}{10}
\pict{imgs/2-hk2.jpeg}{10}


\section*{3. カラムクロマトグラフィー・再結晶 (2019/4/24)}

\subsection*{実験方法}

\subsubsection*{シリカゲルクロマトグラフィーの準備}

\pict{imgs/3-1.jpg}{4.5}

図のようにしてシリカゲルクロマトグラフィーの準備を行う。
\begin{enumerate}
\item まずカラム管に綿を入れガラス棒で軽く押さえる。
\item 海砂を入れて溶媒を注ぐ。
\item ガラス棒を抜いてコックを開く。
\item 溶媒で懸濁したシリカゲルを入れコックを開く。
\item 溶媒の高さとシリカの上面の高さが揃ったらカラム管を叩いてシリカを均一に詰める。
\item 粗生成物を1 mLの塩化メチレンに溶かし、そのうち1滴をTLC用に別の容器にとる。さらに1.0 mLの塩化メチレンで希釈してTLCで反応液の様子を調べる。残りをカラムのシリカの面に駒込ピペットを使い吸着させる。再度1 mLの塩化メチレンで洗い込んで再びシリカの上面に吸着させる。最後に展開溶媒(塩化メチレン:ヘキサン = 2:1)1 mLで洗い込んで再びシリカの上面に吸着させる。
\item 海砂を乗せ、溶媒を上まで注ぐ。

\end{enumerate}
\subsubsection*{カラムクロマトグラフィーによる分取}
準備ができたらコックを開き溶媒を50 mLずつ三角フラスコに分取し、分離の様子をTLCで調べる。

\begin{enumerate}
\item オルト体が溶出するまでは「塩化メチレン:ヘキサン = 2:1」の展開溶媒で溶出する。
\item オルト体が溶出したことを確認したら、溶出溶媒を塩化メチレンに変えて分取する。
\item フェノールや副生成物が溶出したことを確認したら「塩化メチレン:酢酸エチル = 10 :1」に溶出溶媒を変えて分取する。
\item TLCで調べた結果を元にオルト体、フェノール、パラ体の区画をそれぞれまとめて溶媒を濃縮する。オルト体とフェノールは2分、パラ体は10分真空ポンプで乾燥して重さと収率を記録する。
\item 結晶体であるパラ体については融点を記録する。

\end{enumerate}
\subsection*{結果}

\begin{table}[]
\begin{tabular}{|c|c|c|c|c|}
\hline
& 分画         & 様子      & 重さ     & 収率      \\ \hline
オルト体       & 1$\sim$5   & 淡い黄色の液体 & 0.63 g & 23.7 \% \\ \hline
フェノール・副生成物 & 6$\sim$14  & 濃い黄色の液体 & 0.36 g &         \\ \hline
パラ体        & 16$\sim$20 & 黄褐色の固体  & 0.44 g & 16.5 \% 
\\ \hline
\end{tabular}
\end{table}

\begin{itemize}
\item パラ体の融点 : 80~83 ℃, 87~89 ℃

\end{itemize}
\subsubsection*{TLC}

\pict{imgs/3-tlc1.jpeg}{6}
\pict{imgs/3-tlc2.jpeg}{6}
\pict{imgs/3-tlc3.jpeg}{6}


\subsection*{考察}
今回のカラムクロマトグラフィーでは溶出の段階が進むにつれて溶出溶媒を変えていった。これは分離する混合物の極性に合わせて溶媒を選ぶ必要があるからである。はじめから極性が強い溶媒で溶出してしまうと無極性物質も極性物質も同時に溶出されてしまう。極性の小さい溶媒から順に大きくしていくことで、それに見合った極性の物質を分離することができる。したがってはじめは極性の小さいヘキサンを塩化メチレンに混ぜ、後半では極性の大きい酢酸エチルを用いた。4-ヒドロキシベンズアルデヒドは電子求引性のホルミル基と電子供与性のヒドロキシ基が反対側に位置するので極性が強く、一方の2-ヒドロキシベンズアルデヒドは極性が弱い。したがって、2-ヒドロキシベンズアルデヒド→フェノール→4-ヒドロキシベンズアルデヒドの順で溶出した。

今回の呈色で用いたヨウ素発色について調べたところ、有機化合物とヨウ素の相互作用によるもので反応や結合では説明できないということである。またヨウ素が二重結合を切断するので 不飽和脂肪酸を持つものが強く発色する。

\subsection*{課題}
\subsubsection*{(2)-2 TLCのRf値がオルト体>フェノール>パラ体になる理由}
4-ヒドロキシベンズアルデヒドは電子求引性のホルミル基と電子供与性のヒドロキシ基が反対側に位置するので極性が強い。一方2-ヒドロキシベンズアルデヒドは隣り合うシラノール基とヒドロキシ基で分子内水素結合を起こすためにシリカゲルに吸着しにくくなる。したがってRf値がオルト体>フェノール>パラ体となる。


\section*{4. Grignard反応 : 試薬の減圧蒸留 (2019/4/25)}

\subsection*{目的}
芳香族化合物の関与する求核反応であるGrignard反応を行う。水が混在すると反応が進行しない無水反応である。エステルに対して2分子のGrignard試薬が反応する。



\subsection*{実験方法}

ブロモベンゼンを減圧蒸留で50 mL蒸留する。初留と本留の重さを測る。本留にストッパーをしてパラフィルムを巻いて保存する。

\subsubsection*{減圧蒸留}

\begin{enumerate}
\item リービッヒ冷却管に水を流す。
\item ブンゼンバーナーの芯を完全開放する。
\item 真空ポンプにつないだ耐圧ゴム管をを折り曲げておいて、ポンプのスイッチを入れる。
\item マノメーターのコックを開ける。
\item 突沸しないように注意しながら、ゆっくりブンゼンバーナー芯を締めて減圧を開始する。同時にマノメーターの変化を確認する。
\item 減圧度が一定になり、値を読んだらマノメーターのコックを閉じる。
\item 沸点が50度以上になるように減圧を調節し、減圧尾を一定にする。このときの計算沸点は予め求めておく。
\item 油浴の温度を上げ始める。計算した沸点+30℃が目安。
\item はじめから加熱して減圧すると必ず突沸するので注意する。温度計の温度が一定になったらすぐに本留に切り替える。
\item 本流を十分にとり、温度が下がり始めたら、初留のフラスコに変え、油浴を下げて減圧を解除する。その後ポンプのスイッチを切る。


\end{enumerate}

\pict{imgs/4-zu.jpeg}{12}
\subsection*{結果}
減圧蒸留では25 mmHg、60 ℃で沸騰する計算だったが、実際には54 ℃で温度が安定した。

\begin{itemize}
\item 本留の重さ : 51.31 g
\item 初留の重さ : 17.34 g
\item 収率 : $\frac{51.31}{74.55} = 0.6882 \Longrightarrow 68.8\%$
\end{itemize}
\subsection*{考察}
減圧蒸留における沸点の計算は「薬学実習1」の沸点換算図表をもとに計算した。しかし、この表で求められる値はあくまで目安なので実際の沸点とは異なることがある。


\section*{5. Grignard反応 (2019/4/26)}

\subsection*{実験方法}
\begin{enumerate}
\item 器具を熱いうちに組み立てる。Mg turnings 2 g を 500 mL 三角フラスコに入れる。マグネット撹拌子で激しく撹拌を始める。滴下ロートに無水エーテル 25 mL を入れる。
\item 氷水浴を準備して、ジムロート冷却管に水を流した後、ブロモベンゼン13.5 g を15 mLの無水エーテルに溶かし、ストッパーを取って三頚フラスコに一気に加える。反応溶液が白濁したら滴下ロートの無水エーテルを一気に加える。エーテルの沸騰が激しい時は氷水浴につけて冷やすが、冷やしすぎて反応が止まらないように注意する。エーテルがわずかに沸騰している状態で撹拌を続けると淡褐色の反応液を得て、わずかな金属片が残るのみである。最後に反応溶液を氷水で冷やして室温にする。
\item 滴下ロートに5 gの安息香酸メチルを15 mLの無水エーテルに溶かしたものを入れ、室温にした上で調整したグリニャール試薬に緩やかな還流が起きるように20分かけてゆっくり滴下する。滴下が終わったら滴下ロートを外し、ストッパーをつける。
\item 油浴で50~60度にして30分間加熱還流を行う。このとき、油温と油面の高さに注意する。還流を終えたら室温まで放冷し、氷水で冷やす。

\pict{imgs/5-zu.jpeg}{5}

\item 500 mL三角フラスコに 50 mLの10\% v/v 希硝酸を入れ、氷水で冷やしながら反応液に注ぐ。固化した場合はよくスパーテルで砕いてから希硝酸に注ぐ。残ったMgが溶けるのを待ったら水で洗いこむ。
\item 分液ロートに移し、エーテルで洗いこみ、30 mLのエーテルで3回抽出する。エーテル層を飽和食塩水50 mLで洗い、$\rm Na_2SO_4$で脱水したのちに濾過する。
\item 溶媒を留去し、真空ポンプで10分間減圧乾燥したものの重さ・粗収率・色・形状を記録する。TLC用に粗生成物の一部を保存する。
\item トリフェニルメタノールはエーテルに溶けやすく、ヘキサンに溶けにくいので、以下の要領で再結晶を行う。

\end{enumerate}
\subsubsection*{再結晶}
少量のエーテルに粗生成物を加熱して溶かし、ヘキサンを加えて結晶を析出させる。濾取した結晶をヘキサンでよく洗う。再結晶をスパーテル半匙程度取って、保存した粗生成物とともに塩化メチレン1 mLに溶解してTLCで再結晶による生成の度合いを調べる。

\subsection*{結果}
\begin{itemize}
\item 今回用いたブロモベンゼンは13.53 g、安息香酸メチルは4.98 gである。
\item 操作2で無水エーテルを加えると白濁していた液が徐々に褐色へと変化していった。

\end{itemize}
\subsubsection*{粗結晶}
\begin{itemize}
\item 重さ : 8.17 g
\item 粗収率 : 85.8 %
\item 形状・色 : 黄色い湿り気のある結晶
\end{itemize}
\subsubsection*{再結晶}
\begin{itemize}
\item 重さ : 3.83 g
\item 収率 : 40.2 %
\item 形状・色 : 白いサラサラとした結晶
\item 融点 : 139~142 ℃

\end{itemize}
\subsubsection*{TLC}
\pict{imgs/4-tlc.jpeg}{4.5}

\subsection*{考察}
今回の反応では安息香酸メチル1 molに対して、ブロモベンゼンが2 mol 反応する。ブロモベンゼンは安息香酸メチルの2倍以上あるので収率を求めるときは安息香酸メチルの物質量をもとに算出する。

TLCではビフェニルの方がRf値が大きい。これはビフェニルがシリカゲルと引き合う基を持っていないのに対して、トリフェニルメタノールはヒドロキシ基を持っているからだと考えられる。

\subsection*{課題}

\subsubsection*{(3) - 1 グリニャール反応におけるエーテルの必要性とグリニャール反応の反応機構}

\pict{imgs/4-1.jpeg}{10}

グリニャール反応において、エーテルは無水状態を保つのに必要であることに加えて、図のようにグリニャール試薬と錯体を形成し、グリニャール試薬を溶かすという意味合いを持つ。

\pict{imgs/4-2.jpeg}{4}


\subsubsection*{(3) - 2 反応系中に水が混入したときに起こる反応}

グリニャール試薬は求核剤であると同時に強力な塩基でもあり、以下のように反応する。

\pict{imgs/4-3.jpeg}{10}

\subsubsection*{(3) -3 ビフェニル生成の反応機構}

グリニャール試薬を生成の際に以下ようなフェニルラジカルを経由する
\pict{imgs/4-4.jpeg}{10}
フェニルラジカル同士が反応するとビフェニルが生成する
\pict{imgs/4-5.jpeg}{7.5}

\subsubsection*{(3) - 4 エステルからグリニャール試薬でケトンを合成する方法}
カルボニル炭素の求電子性エステルよりケトンの方が強いため、グリニャール試薬をエステルに作用させると反応がケトンで止まらない。
\pict{imgs/4-6.jpeg}{10}
そこで以下のようにエステルを求電子性の高い酸塩化物へと変換する。
\pict{imgs/4-7.jpeg}{10}
これに一等量のグリニャール試薬を加えれば、試薬が酸塩化物と優先的に反応してケトンで反応が止まる。
\pict{imgs/4-8.jpeg}{10}

\section*{6. トリフェニルメタノールの反応  (2019/05/07)}
\subsection*{実験方法}
\begin{enumerate}
\item 0.99 g のトリフェニルメタノールを広口100 mLナスフラスコに測りとり、乾いた100mLメスシリンダーに測りとった 25 mLのメタノールに加熱して溶かし、別の三角フラスコにとった1 mL の47 %HBr水溶液を加えて80度以上の湯浴で15分間ジムロート冷却管をつけて緩やかに加熱する。
\item 氷水で冷やして反応容器内壁をスパーテルでこすって結晶化させる。
\item 析出した結晶を濾取し、冷えたメタノールで洗う。
\item 真空ポンプで10分間乾燥して重さと融点を測定する。
\end{enumerate}
\subsection*{結果}
\subsubsection*{得られた結晶}
\begin{itemize}
\item 重さ : 0.78 g
\item 色 : 白
\item 形状 : 粉末状でふわふわとしている
\item 融点 : 78~85 ℃, 82~86 ℃
\item 氷冷してもなかなか結晶化しなかったが、スパーテルで擦っていると急に析出した



\end{itemize}
\subsection*{考察}
\begin{itemize}
\item 生成物の融点を測定し構造決定を行い、反応機構を考える

得られた物質の融点が低いことから、生成物の候補は以下の三つに絞られた。
\[\rm (C_6H_5)_3COOCCH_3  \ \ \ \ \   (C_6H_5)_3CH \ \ \ \ \ (C_6H_5)_3COCH_3\]

自分たちの実験では酢酸などを用いてないことから$\rm (C_6H_5)_3COOCCH_3$が候補から外され、$\rm (C_6H_5)_3CH$についてもトリフェニルメチルカチオンに$\rm H^+$が結合するのは難しいと考えられるので、$\rm (C_6H_5)_3COCH_3$が生成物であると推測される。
\end{itemize}
\subsection*{課題}
\subsubsection*{(4) 反応a,b,cの反応機構を書く}
\begin{itemize}
\item (b) 生成物 : $p- \rm (C_6H_5)_2CHC_6H_4C(C_6H_5)_3$
立体障害が大きいので2つが上のように反応することはない。電子が非局在化することで以下のような反応が起こる。

\pict{imgs/6-1.jpeg}{7}
\pict{imgs/6-2.jpeg}{10}

\item (a) 生成物 : $\rm (C_6H_5)_3C-Br$
\item (c) 生成物 : $\rm (C_6H_5)_3COCH_3$
(a)と(c)では同じHBr水溶液を加えているが、生成物が異なる。
(a)ではトリフェニルメチルカチオンに対して$\rm Br^-$が作用すると$\rm (C_6H_5)_3C-Br$が生成する。一方(c)では$\rm Br^-$と同時にメタノールが作用する。$\rm Br^-$は求核性が強いと同時に脱離能も高いのに対し、メタノールが一度作用して$\rm (C_6H_5)_3COCH_3$が生成すると元に戻ることはほとんどない。$\rm (C_6H_5)_3C-Br$よりも$\rm (C_6H_5)_3COCH_3$で熱力学的に安定である。したがって(c)では最終的には$\rm (C_6H_5)_3COCH_3$が生成される。

\pict{imgs/6-3.jpg}{9}

\end{itemize}

\end{document}
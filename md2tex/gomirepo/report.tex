\documentclass[a4paper,papersize,dvipdfmx]{jsarticle}
\usepackage{ascmac}
\usepackage{mathtools, amssymb,bm}
\usepackage{comment}
\usepackage[hiresbb]{graphicx}
\usepackage{tcolorbox,color}
\usepackage{here}
\tcbuselibrary{raster,skins,breakable}

\newcommand{\pic}[1]{\begin{center} \includegraphics[width=1.0\linewidth,clip]{#1} \end{center}}   %写真用
\newcommand{\pict}[2]{\begin{center} \includegraphics[width= {#2} cm]{#1} \end{center}}   %写真用
\newcommand{\piccap}[3]{\begin{figure}[H] \centering \includegraphics[width= {#2} cm]{#1} \caption{#3} \label{fig {#1}} \end{figure}} %キャプションつき画像
\newcommand{\redunderline}[1]{\textcolor{red}{\underline{¥textcolor{black}{#1}}}}   %赤いアンダーライン
\newcommand{\mon}[1]{\item[({#1})] \ }
\newcommand{\ctext}[1]{\raise0.2ex\hbox{\textcircled{\scriptsize{#1}}}}%文字を丸囲みする(2桁の数字までならいける)

% 画像を貼る時はjpgかjpegで、pngはうまくいかない時もある

%\itemを四角で囲った数字にする場合は以下のコメントアウトを消す
%\renewcommand{\labelenumi}{\textbf{\framebox[1.5zw]{\theenumi}}}


%enumerateの2階層めのカウンタを1,2,3, にする時は以下のコメントアウトを消す
\renewcommand{\theenumii}{\arabic{enumii}}

%enumerateのカウンタについては以下を参照
% http://www3.otani.ac.jp/fkdsemi/pLaTeX_manual/kajyo.html


%enumerateの番号の出力形式を変更するには、カウンタの値を出力する命令を定義し直す。
%レベル	カウンタ	出力する命令	デフォルトの出力
%1	enumi	¥theenumi	アラビア数字(1,2,3,・・・)
%2	enumii	¥theenumii	小文字のアルファベット(a,b,c,・・・)
%3	enumiii	¥theenumiii	小文字のローマ数字(小文字のローマ数字(ⅰ,ⅱ,ⅲ,・・・)
%4	enumiv	¥theenumiv	大文字のアルファベット(A,B,C,・・・)
%例:¥enumiカウンタを大文字のローマ数字で出力する設定
% ¥renewcommand{¥theenumi}{¥Roman{enumi}}

% 番号の出力形式
%命令	出力形式
%¥arabic	アラビア数字(1、2、3、・・・)
%¥roman	ローマ数字(ⅰ、ⅱ、ⅲ、・・・)
%¥Roman	ローマ数字(Ⅰ、Ⅱ、Ⅲ、・・・)
%¥alph	アルファベット(a、b、c、・・・)
%¥Alph	アルファベット(A、B、C、・・・)




\begin{document}

\title{生物物理学レポート}
\author{10191043 鈴木健一}
%作成日を入れる場合は消す
\date{}
\maketitle

%以下の3つからフォントサイズを選択するとよい
%\footnotesize
%\small
%\normalsize
\large

\vspace*{40pt}

\section*{配布資料のまとめ}

\

タンパク質結晶学は技術的の発展によって分解能の向上が進むとともに発展を遂げてきた。
X線による構造解析が原子構造の決定に十分な能力があることが証明され、
分子量の大きいタンパク質や酵素-基質複合体、ウイルス粒子、
膜タンパク質といった比較的複雑なタンパク質の構造が明らかとなった。
これからはより大きなタンパク質の構造解析にスポットが当てられるだろう。

今後のタンパク質結晶学の発展における課題として分解能の向上が求められている。
多くの結晶構造は2.0〜2.5Åの範囲で解析されているが、
水素原子の配置を求めるには1.2Å以上の分解能が必要である上に、
それでも電子雲の形状を求めるには不十分だからである。

そこで第三世代のシンクロトロン光源や自由電子X線レーザーといった強い光源を用いたり、
低温電子顕微鏡法や核磁気共鳴法と組み合わせたりするなどの工夫により、様々なタンパク質の構造が明らかにされてきた。

タンパク質の構造を解明することは生命現象の根源を知ること、進化の歴史を辿ることに繋がる。
そのために今では、タンパク質構造の重要性が結晶学以外の分野でも注目されている。

タンパク質結晶学が解明すべき課題はまだまだ多く残されている。

\vspace*{40pt}

\section*{考察および立体構造解析が望まれるタンパク質について}

\

タンパク質の構造を知ることは学問的に重要なだけでなく、
産業や我々の生活の発展にも大きく関わっている。

一般的に我々が飲む薬は体内のタンパク質に結合して、
タンパク質の働きを調節することで病気を治したり症状を和らげたりする。
つまり、病因となるタンパク質の働きや機序を理解することで
効果的な薬品を設計することができる。
この時に注目するのがタンパク質と酵素の特異性である。
標的となるタンパク質に特異的に結合する能力が高いほど薬の無駄がなく副作用も抑えられると言えるだろう。
そのためにタンパク質の構造を詳しく解析し、
特異的にタンパク質に結合する酵素や化合物を設計するヒントを得ることが重要となってくる。

医薬品だけでなく洗剤や食品製造においても酵素が大きな役割を果たしている。そのような産業では構造解析によって得られた発見の恩恵をダイレクトに受けることは容易に想像できる。

近年の人口増加を受けて世界では人工肉の研究が進んでいる。
イスラエルの企業では2021年に商用化を目指しているところもある。
家畜を育てるよりもローコストで衛生的、栄養的にも利点が見込まれる点で画期的であるが、
まだ食用として十分に仕上がった人工肉は生産されていない。
噛みごたえのあるステーキ肉は、筋肉、組織、脂肪や血管細胞の比率が重要となってくる。
牛や豚など、食用の動物のタンパク質の解析によって、
組織構成のメカニズムや比率の整った肉を作るのに必要な情報が得られるだろう。
いずれは人工肉が我々の食卓に出るのが当たり前になる日が来るかもしれない。

\vspace*{40pt}

\section*{授業の感想}

\

薬学実習でもX線構造解析をはじめとする様々な解析法を用いてタンパク質の情報を調べた。この授業と並行して実習に取り組むことでより深くタンパク質の構造や結晶の解析についての理解を得ることができた。複雑な構造から苦手意識を持っていたが、授業と実習を経験してタンパク質の構造に親しみを持つことができた。


\end{document}

\documentclass[a4paper,papersize,dvipdfmx]{jsarticle}
\usepackage{ascmac}
\usepackage{mathtools, amssymb,bm}
\usepackage{comment}
\usepackage[hiresbb]{graphicx}
\usepackage{tcolorbox,color}
\usepackage{here}
\usepackage{multirow} % 表でセルを結合しているときに必要
\usepackage{booktabs} % 表の形式をbooktabにすると必要
\usepackage{listings}
\tcbuselibrary{raster,skins,breakable}

\newcommand{\pic}[1]{\begin{center} \includegraphics[width=1.0\linewidth,clip]{#1} \end{center}}   %写真用
\newcommand{\pict}[2]{\begin{center} \includegraphics[width= {#2} cm]{#1} \end{center}}   %写真用
\newcommand{\piccap}[3]{\begin{figure}[H] \centering \includegraphics[width= {#2} cm]{#1} \caption{#3} \label{fig {#1}} \end{figure}} %キャプションつき画像
\newcommand{\redunderline}[1]{\textcolor{red}{\underline{¥textcolor{black}{#1}}}}   %赤いアンダーライン
\newcommand{\mon}[1]{\item[({#1})] \ }
\newcommand{\ctext}[1]{\raise0.2ex\hbox{\textcircled{\scriptsize{#1}}}}%文字を丸囲みする(2桁の数字までならいける)

%ここからソースコードの表示に関する設定
% 参考 https://qiita.com/ta_b0_/items/2619d5927492edbb5b03
\lstset{
  basicstyle={\ttfamily},
  identifierstyle={\small},
  commentstyle={\smallitshape},
  keywordstyle={\small\bfseries},
  ndkeywordstyle={\small},
  stringstyle={\small\ttfamily},
  frame={tb},
  breaklines=true,
  columns=[l]{fullflexible},
  numbers=left,
  xrightmargin=0zw,
  xleftmargin=3zw,
  numberstyle={\scriptsize},
  stepnumber=1,
  numbersep=1zw,
  lineskip=-0.5ex
}

% ソースコードを入れる方法
%\begin{lstlisting}[caption=hoge,label=fuga]
% #include<stdio.h>
% int main(){
%    printf("Hello world!");
% }
% \end{lstlisting}


% 複数図を横に並べるときのヒント http://wright.mydns.jp/?p=704
% \begin{figure}{H}
% \centering
% \begin{minipage}{0.22\hsize}
% \piccap{}{}{}
% \end{minipage}
% \begin{minipage}{0.06\hsize}
% \hspace{2mm}
% \end{minipage}
% \begin{minipage}{0.22\hsize}
% \piccap{}{}{}
% \end{minipage}
% \end{figure}

% 画像を貼る時はjpgかjpegで、pngはうまくいかない時もある

%\itemを四角で囲った数字にする場合は以下のコメントアウトを消す
%\renewcommand{\labelenumi}{\textbf{\framebox[1.5zw]{\theenumi}}}


%enumerateの2階層めのカウンタを1,2,3, にする時は以下のコメントアウトを消す
\renewcommand{\theenumii}{\arabic{enumii}}

%enumerateのカウンタについては以下を参照
% http://www3.otani.ac.jp/fkdsemi/pLaTeX_manual/kajyo.html


%enumerateの番号の出力形式を変更するには、カウンタの値を出力する命令を定義し直す。
%レベル	カウンタ	出力する命令	デフォルトの出力
%1	enumi	¥theenumi	アラビア数字(1,2,3,・・・)
%2	enumii	¥theenumii	小文字のアルファベット(a,b,c,・・・)
%3	enumiii	¥theenumiii	小文字のローマ数字(小文字のローマ数字(ⅰ,ⅱ,ⅲ,・・・)
%4	enumiv	¥theenumiv	大文字のアルファベット(A,B,C,・・・)
%例:¥enumiカウンタを大文字のローマ数字で出力する設定
% ¥renewcommand{¥theenumi}{¥Roman{enumi}}

% 番号の出力形式
%命令	出力形式
%¥arabic	アラビア数字(1、2、3、・・・)
%¥roman	ローマ数字(ⅰ、ⅱ、ⅲ、・・・)
%¥Roman	ローマ数字(Ⅰ、Ⅱ、Ⅲ、・・・)
%¥alph	アルファベット(a、b、c、・・・)
%¥Alph	アルファベット(A、B、C、・・・)

% ページ番号を消す場合は以下のコメントアウトを消す
%\pagestyle{empty}

\begin{document}

\title{細胞情報学教室}
\author{10191043 鈴木健一}
%作成日を入れる場合は消す
\date{}
\maketitle

%以下の3つからフォントサイズを選択するとよい
%\footnotesize
%\small
%\normalsize


\part*{ストレス応答の分子機構}

\section*{目的}
ストレス応答の分子機構を解析する手法の一つとして、
小胞体ストレスに応答したmRNAのスプライシングおよび転写誘導を検討する。
また、同時に小胞体ストレス誘導刺激による細胞死を検討する。
これらの実験操作を通して細胞レベルでのRNAの解析手法および細胞死の検出方法についての理解を深める。

\section*{実験方法}
実習書に基づきTAの指示にしたがって実験を行った。
ただし、BiPのPCRにおいてサイクル数を16,19,24から
それぞれ19,24,30に変更している。


\section*{結果・考察・課題}
今回は課題の内容に則しながら結果と考察を行う。

\subsection*{単離したRNAの品質}

以下の画像からA,B,Cそれぞれにおいてメジャーバンドが3本ずつ観察され、上から順に28S,18S,5Sとなっている。
その中で目的とするrRNAのバンドは28Sと18Sである。
本来28Sと18Sの質量比は2:1であり、28Sの方が明瞭なバンドが出るはずだが、
28SのRNAは細胞内でプロセシングを受けると18SのRNAと同じような泳動度を示す。
したがって画像では18Sの方が明瞭なバンドが現れているが、
28Sのバンドも十分に現れているので、RNAの品質は問題ないことが確認された。

\piccap{images/rna.jpg}{7}{RNAの電気泳動}


\subsection*{Xbp1のスプライシング}
関接的に撮影した画像ではよくわからないが、うっすらとバンドを確認することができた。
しかし、PtsI切断部位とステムループ構造までを確認することはできなかった。
\piccap{images/xbp1.jpg}{7}{Xbp1の電気泳動}
本来は次の画像のようにインキュベートの前後でバンドの位置の変化が確認できた。
\pict{images/good_xbp1.jpg}{7}

\subsection*{BiP mRNAの発現量}
電気泳動の結果、RPS18のバンドだけが見られ、BiPのバンドが全く観察できなかった。
\piccap{images/bip.jpg}{7}{BiPの電気泳動}
そのため、昨年までの模範結果をもとに考察を行う。
\pict{images/good_bip.jpg}{7}
RPS18はハウスキーピング遺伝子由来で定常的に発現するタンパク質であるため、
このバンドの強さが一定である限り、サンプル間でのタンパク質の量や濃度に違いがないことが保証できる。
模範の結果ではAよりもB、BよりもCの方がサイクル数が少ない時のバンドが明瞭になっている。
このことから、もともとのBiP mRNA量はC$>$B$>$Aとなっていることがわかる。

\subsection*{DTTの濃度と時間による生存率}

以下の表のように30分間の刺激では、刺激がない群とほとんど生存率に差がなかった。
一方、3時間ストレス下にあった群では、有意に生存率が低下した。

\begin{table}[H]
\centering
\begin{tabular}{@{}ccccc@{}}
\toprule
DTT &  None & 10mM 30min & 10mM 180min & 50mM 180min \\ \midrule
生存率(\%) & 96.5 & 95.1           & 79.0            & 70.3            \\ \bottomrule
\end{tabular}
\end{table}


\subsection*{総合課題}
\begin{enumerate}
\mon{1}

XBP1タンパク質は常時mRNAから発現されているため、急なストレスに対しても素早く応答できるという利点がある。

\mon{2}

スプライス型、非スプライス型それぞれに特異的なプライマーを作るには、
スプライシングされる部分を跨がるようにそれぞれの塩基配列を持ったプライマーを作成すれば良い。

\mon{3}

ゲノム除去反応を行わなかった場合、cDNAだけでなく微量に含まれるgDNAも増幅されてしまう。
したがってgDNAでは間にイトロンがコードされているcDNAのエキソン間の部分をまたがるようなプライマーを用いれば良い。

\mon{4}

前半よりも実習終了時刻が早くてよかった。TAが班に一人いるのでわからないこともすぐに聞けるところが心強い。

\end{enumerate}

\end{document}

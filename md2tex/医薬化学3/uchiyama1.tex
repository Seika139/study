\documentclass[a4paper,papersize,dvipdfmx]{jsarticle}
\usepackage{ascmac}
\usepackage{mathtools, amssymb,bm}
\usepackage{comment}
\usepackage[hiresbb]{graphicx}
\usepackage{tcolorbox,color}
\usepackage{here}
\usepackage{multirow} % 表でセルを結合しているときに必要
\usepackage{booktabs} % 表の形式をbooktabにすると必要
\usepackage{listings}
\tcbuselibrary{raster,skins,breakable}

\newcommand{\pic}[1]{\begin{center} \includegraphics[width=1.0\linewidth,clip]{#1} \end{center}}   %写真用
\newcommand{\pict}[2]{\begin{center} \includegraphics[width= {#2} cm]{#1} \end{center}}   %写真用
\newcommand{\piccap}[3]{\begin{figure}[H] \centering \includegraphics[width= {#2} cm]{#1} \caption{#3} \label{fig {#1}} \end{figure}} %キャプションつき画像
\newcommand{\redunderline}[1]{\textcolor{red}{\underline{¥textcolor{black}{#1}}}}   %赤いアンダーライン
\newcommand{\mon}[1]{\item[({#1})] \ }
\newcommand{\ctext}[1]{\raise0.2ex\hbox{\textcircled{\scriptsize{#1}}}}%文字を丸囲みする(2桁の数字までならいける)

%ここからソースコードの表示に関する設定
% 参考 https://qiita.com/ta_b0_/items/2619d5927492edbb5b03
\lstset{
  basicstyle={\ttfamily},
  identifierstyle={\small},
  commentstyle={\smallitshape},
  keywordstyle={\small\bfseries},
  ndkeywordstyle={\small},
  stringstyle={\small\ttfamily},
  frame={tb},
  breaklines=true,
  columns=[l]{fullflexible},
  numbers=left,
  xrightmargin=0zw,
  xleftmargin=3zw,
  numberstyle={\scriptsize},
  stepnumber=1,
  numbersep=1zw,
  lineskip=-0.5ex
}

% ソースコードを入れる方法
%\begin{lstlisting}[caption=hoge,label=fuga]
% #include<stdio.h>
% int main(){
%    printf("Hello world!");
% }
% \end{lstlisting}


% 複数図を横に並べるときのヒント http://wright.mydns.jp/?p=704

% \begin{figure}[H]
% \begin{center}
% \begin{tabular}{c}
%
% 一行目
% \begin{minipage}{0.22\hsize}
% \piccap{}{}{}
% \end{minipage}
%
% 列の仕切り
% \begin{minipage}{0.06\hsize}
% \hspace{2mm}
% \end{minipage}
%
% \begin{minipage}{0.22\hsize}
% \piccap{}{}{}
% \end{minipage}
%
% 行間スペース
% \\
% \begin{minipage}{0.06\hsize}
% \vspace{10mm}
% \end{minipage}
% \\
%
% \end{tabular}
% \end{center}
% \end{figure}

% 画像を貼る時はjpgかjpegで、pngはうまくいかない時もある

%\itemを四角で囲った数字にする場合は以下のコメントアウトを消す
%\renewcommand{\labelenumi}{\textbf{\framebox[1.5zw]{\theenumi}}}


%enumerateの2階層めのカウンタを1,2,3, にする時は以下のコメントアウトを消す
\renewcommand{\theenumii}{\arabic{enumii}}

%enumerateのカウンタについては以下を参照
% http://www3.otani.ac.jp/fkdsemi/pLaTeX_manual/kajyo.html


%enumerateの番号の出力形式を変更するには、カウンタの値を出力する命令を定義し直す。
%レベル	カウンタ	出力する命令	デフォルトの出力
%1	enumi	¥theenumi	アラビア数字(1,2,3,・・・)
%2	enumii	¥theenumii	小文字のアルファベット(a,b,c,・・・)
%3	enumiii	¥theenumiii	小文字のローマ数字(小文字のローマ数字(ⅰ,ⅱ,ⅲ,・・・)
%4	enumiv	¥theenumiv	大文字のアルファベット(A,B,C,・・・)
%例:¥enumiカウンタを大文字のローマ数字で出力する設定
% ¥renewcommand{¥theenumi}{¥Roman{enumi}}

% 番号の出力形式
%命令	出力形式
%¥arabic	アラビア数字(1、2、3、・・・)
%¥roman	ローマ数字(ⅰ、ⅱ、ⅲ、・・・)
%¥Roman	ローマ数字(Ⅰ、Ⅱ、Ⅲ、・・・)
%¥alph	アルファベット(a、b、c、・・・)
%¥Alph	アルファベット(A、B、C、・・・)

% ページ番号を消す場合は以下のコメントアウトを消す
%\pagestyle{empty}

\begin{document}

\title{分子軌道論}
\author{SUZUKEN}
%作成日を入れる場合は消す
\date{}
\maketitle

%以下の3つからフォントサイズを選択するとよい
%\footnotesize
%\small
%\normalsize

\section{水素の分子軌道}
すべての化合物の分子軌道の基礎となるのは水素分子の軌道である。水素の分子軌道は水素原子軌道の線形結合
[LCAO (linear combination of atomic orbitals)](一次結合ともいう)によってつくられる。
すなわち、結合性軌道$\psi_1$は$\phi_1+\phi_2$であり、反結合性軌道$\psi_2$は$\phi_1-\phi_2$である。実際の軌道では、原子軌道$\phi$に係数$c$が掛けられ、 $c1^2+c_2^2=1$になるように規格化されている。

\[
  \psi_1 = c_1 \psi_1 + c_2 \psi_2
\]
\[
  \psi_2 = c_1 \psi_1 - c_2 \psi_2
\]

係数の二乗は電子の存在確率を表すので、その合計値は当然1になる。
水素分子の場合、$c_1$と $c_2$は同じだから0.707になる。
化学反応を議論する際は、s軌道は球形であることから円で表す。
水素分子軌道は下記のように図示することができる。

\pict{image/suiso.png}{12}

水素の原子軌道2個から水素分子軌道ができる際、結合性軌道と反結合性軌道が生成する。
上図では、結合性軌道は白と白で位相が合った重なり、反結合性軌道は白と黒で位相が合わない重なりである。
原子軌道φ1(φ2)において矢印で示した電子は対を作り、水素分子軌道の結合性軌道ψ1に入る。

結合性軌道と反結合性軌道の2個の軌道ができる理由は、次のように考える。
水素の原子軌道において収容される最大電子数は2個である。
2個の水素原子で は2×2個であるから、分子軌道においても同じ数の電子を収容するだけの軌道が存在することによる。
すなわち、4/2で2個である。
数学的には2次関数を 解くので、解が2個存在することに符合する。
水素原子に限らず、軌道を図示する際は結合性軌道を下側に、反結合性軌道を上側に書く。
その際、軌道のエネルギーは紙面の上の方に行くほど高く(不安定)なる。
定性的議論では円の大きさは適当に書いてもよい。

\section{フロンティア軌道論}
電子の波動性を考慮した量子化学に基づく反応論のこと。
最高被占軌道(HOMO)と最低空軌道(LUMO)が反応に関わると考えます。HOMOは電子が入っている軌道の中で最もエネルギーが高い軌道であり、
LUMOは電子が入っていない軌道の中でエネルギーが最も低い軌道です。このHOMOとLUMOを合わせてフロンティア軌道(frontier orbital)と呼びます。

\pict{image/00.png}{8}

フロンティア軌道論の主張は、「求核試薬のHOMOと求電子試薬のLUMOのうち、それぞれの軌道の広がりが最も大きい部分が反応点になる」というものです。
つまり、反応は単に電子が多い・少ないだけで決まるのではなく、電子の軌道によって決まるということです。

なお、「フロンティア(frontier)」は「境」や「前線」を意味しますが、
この理論がこのように名付けられているのは「エネルギー的に境目の2つの軌道(HOMOとLUMO)が反応に関わる」ということによります。

\subsection{反応生が大きくなる条件}

\subsubsection{1. 軌道の広がりが大きい}
軌道の広がりというのは少しあやふやな言葉であり、正しくは「電子の分布確率が高い空間が広い」ということです。
「軌道」というのは日本の歴史的な使い方ですので、オービタル(orbital)という言葉を今後は使います(orbitalはorbitっぽいものという意味)。

電子の密度分布が広がっているほど、HOMOとLUMOのオービタルが重なりやすく、HOMOの電子がLUMOに流れ込みやすくなります。
エネルギー的な距離と空間的な距離が近いことで、HOMOの電子はLUMOに流れ込みやすくなります。

\subsubsection{2. 位相が同じである}
結合が生じるにはHOMOとLUMOが重なるだけでなく、さらにオービタルの位相が同じでなくてはなりません。
これは電子の波動性に基づきます。位相が同じだと強め合って結合します。反対に位相が異なると弱めあって反結合性となります。

\subsection{ナフタレンの反応}

求電子試薬はLUMOと反応し、求核試薬はHOMOと反応するが、
ナフタレンはどちらの場合でもも$\beta$位に比べて$\alpha$位で置換反応しやすい。
これは電子密度で反応性を考える電子論では説明できない。
$\alpha$位、$\beta$位は以下の通り。

\pict{image/01.png}{5}

等電子密度表面の値を0.0800としてやると、HOMOとLUMOは以下のようになる。

\begin{figure}[H]
\begin{center}
\begin{tabular}{c}

\begin{minipage}{0.22\hsize}
\piccap{image/02.png}{5}{HOMO}
\end{minipage}

\begin{minipage}{0.10\hsize}
\hspace{2mm}
\end{minipage}

\begin{minipage}{0.22\hsize}
\piccap{image/lumo.png}{5}{LUMO}
\end{minipage}

\end{tabular}
\end{center}
\end{figure}

どちらにおいても軌道の広がりは$\beta$位よりも$\alpha$位において大きい。
HOMOとLUMOの広がりが大きいので。どちらの場合でもも$\beta$位に比べて$\alpha$位で置換反応しやすい。

\section{熱力学的支配と速度論的支配}

さて、上で「$\alpha$位の方が反応性が高い」と言いましたが、ここで注意しなければならないことは、
フロンティア軌道論は速度論(kinetics)的、すなわち反応のし易さを示すだけであり、
熱力学(thermodynamics)的な安定性を示すものではない、ということです。
なので、ナフタレンの場合、「いかなる条件下でもα位で全て反応するわけではない」のです。
生成物の中には勿論、$\beta$位置換生成物も含まれるし、反応条件を変えれば$\beta$位置換生成物の方が主生成物となることもあります。

一般に低温・短時間の条件下では速度論が優先され、速度論生成物(kinetic product)が主生成物となります。
一方、高温・長時間の条件下では熱力学が優先され、熱力学生成物(thermodynamics product)が主生成物となります。
これらが一致することもありますが、今回は異なります。
したがって低温下ではα位置換生成物が主生成物となり、高温下では$\beta$位置換生成物が主生成物となるのです。
分かりづらいと思うので、エネルギー関係のグラフを以下に示します。

\pict{image/shihai.png}{7}

\section{分子軌道のフロンティア軌道を考える}

分子軌道 (Molecular orbital)は、原子軌道の波の重なりによって生成する。
したがって、分子軌道は当然「波」であり、固有のエネルギーを持っている。
もっとも安定な軌道は、両端を波節(静止点)とする1/2波長、2番目の波は1波長、3番目の波は3/2波長の波である。
物理学でいう「定常波」、すなわち両端を固定した弦の振動を考えればよい。
電子は密閉された箱の中を定常波として運動していると見なすことができる。

一つの軌道にはスピン量子数を異にする2個の電子しか入ることができない。
一般に、それぞれ上向きと下向きの矢印で表す。分子は複数の原子が電子によって結びつけられている。
その際、原子間結合1個は2個の電子で結びつけられている。
原子と原子を結びつけるために必要な価電子の総数を2で割った数の軌道 が電子の詰まった軌道であり、「被占軌道」とよばれる。
それ以外の軌道は電子の入っていない空の軌道であり、「空軌道」とよばれる。

\piccap{image/wave.png}{12}{波の節の数が少ないほど安定な軌道である}

\section{電子論では説明できない現象を軌道論で説明する}
電子論では、エチレンの二量化でシクロブタンが生成することが予想できる。
電子の偏りが期待できない対称構造の中性分子に強引に正負の電荷を作り、
相互作用させれば結合ができると考えるわけである。
曲がった矢印の理論で考えてもよい。
ところが、フロンティア軌道論による予想ではいくら加熱してもHOMOとLUMOの位相が合わないので生成しない。

\pict{image/04.png}{12}

\section{ウッドワード・ホフマン則}

\begin{tcolorbox}
相互作⽤系の分⼦軌道の対称性が、反応の全過程を通じて不変であり保存される。
軌道対称性保存則ともいう。
\end{tcolorbox}


電⼦環状反応、付加環化反応などにおいて、これまでの有機電⼦論では説明できなかった反応の⽴体選択性を⾒事に説明できる。
例えば、電⼦環状反応では、共役電⼦系の⻑さと反応条件(熱反応、光反応)で⽴体反応選択性に以下の関係性がある。

\begin{table}[H]
\centering
\begin{tabular}{|c|c|c|}
\hline
$\pi$電子数 & 熱反応 & 光反応 \\ \hline
4n & 同旋的 & 逆旋的 \\ \hline
4n+2     & 逆旋的 & 同旋的 \\ \hline
\end{tabular}
\end{table}

\section{参考サイト}
\begin{itemize}
\item https://sites.google.com/site/harano2011/yuuki-kidou-ron/kidou-no-sakuzu
\end{itemize}

\end{document}

\documentclass[a4paper,papersize,dvipdfmx]{jsarticle}
\usepackage{ascmac}
\usepackage{mathtools, amssymb,bm}
\usepackage{comment}
\usepackage[hiresbb]{graphicx}
\usepackage{tcolorbox,color}
\usepackage{here}
\tcbuselibrary{raster,skins,breakable}

\newcommand{\pic}[1]{\begin{center} \includegraphics[width=1.0\linewidth,clip]{#1} \end{center}}   %写真用
\newcommand{\pict}[2]{\begin{center} \includegraphics[width= {#2} cm]{#1} \end{center}}   %写真用
\newcommand{\piccap}[3]{\begin{figure}[H] \centering \includegraphics[width= {#2} cm]{#1} \caption{#3} \label{fig {#1}} \end{figure}} %キャプションつき画像
\newcommand{\redunderline}[1]{\textcolor{red}{\underline{¥textcolor{black}{#1}}}}   %赤いアンダーライン
\newcommand{\mon}[1]{\item[({#1})] \ }
\newcommand{\ctext}[1]{\raise0.2ex\hbox{\textcircled{\scriptsize{#1}}}}%文字を丸囲みする(2桁の数字までならいける)

% 画像を貼る時はjpgかjpegで、pngはうまくいかない時もある

%\itemを四角で囲った数字にする場合は以下のコメントアウトを消す
%\renewcommand{\labelenumi}{\textbf{\framebox[1.5zw]{\theenumi}}}


%enumerateの2階層めのカウンタを1,2,3, にする時は以下のコメントアウトを消す
\renewcommand{\theenumii}{\arabic{enumii}}

%enumerateのカウンタについては以下を参照
% http://www3.otani.ac.jp/fkdsemi/pLaTeX_manual/kajyo.html


%enumerateの番号の出力形式を変更するには、カウンタの値を出力する命令を定義し直す。
%レベル	カウンタ	出力する命令	デフォルトの出力
%1	enumi	¥theenumi	アラビア数字(1,2,3,・・・)
%2	enumii	¥theenumii	小文字のアルファベット(a,b,c,・・・)
%3	enumiii	¥theenumiii	小文字のローマ数字(小文字のローマ数字(ⅰ,ⅱ,ⅲ,・・・)
%4	enumiv	¥theenumiv	大文字のアルファベット(A,B,C,・・・)
%例:¥enumiカウンタを大文字のローマ数字で出力する設定
% ¥renewcommand{¥theenumi}{¥Roman{enumi}}

% 番号の出力形式
%命令	出力形式
%¥arabic	アラビア数字(1、2、3、・・・)
%¥roman	ローマ数字(ⅰ、ⅱ、ⅲ、・・・)
%¥Roman	ローマ数字(Ⅰ、Ⅱ、Ⅲ、・・・)
%¥alph	アルファベット(a、b、c、・・・)
%¥Alph	アルファベット(A、B、C、・・・)




\begin{document}

\title{薬学実習4 生理化学教室}
\author{10191043 鈴木健一}
%作成日を入れる場合は消す
\date{}
\maketitle

%以下の3つからフォントサイズを選択するとよい
%\footnotesize
%\small
%\normalsize

\part*{実験1 細胞内情報伝達の可視化と定量}

\section*{目的}
細胞のシグナル応答を定量的に評価する一連のプロセスを習得する。

細胞外のリガンド(TNF$\alpha$)に応答し細胞質から核に移行するNF$\kappa$Bの挙動を可視化・定量する。

\begin{enumerate}
\item 抗体とHoechstを用いたNF$\kappa$Bと核の染色

\item 画像解析ソフトImageJを用いた顕微鏡画像の処理と定量

\end{enumerate}
\section*{実験方法}
実習書に基づきTAの指示にしたがって実験を行った。

\section*{結果・考察}
TNF$\alpha$を入れなかった時,0.5ng/ml,10ng/ml入れた時の
染色された核のグレースケール画像は以下のようになった。
\pict{images/NoTNFa.png}{4}
\pict{images/TNFa_05ng.png}{4}
\pict{images/TNFa_10ng.png}{4}

これらの画像をもとにTNF$\alpha$の濃度によってNF$\kappa$Bの輝度が変化するかどうかを測定する。

まず濃度に応じて変化するNF$\kappa$Bの輝度を箱ひげ図でまとめた。
\pict{images/graph1.png}{8}
グラフによると平均値や全体的な輝度も濃度の増加に応じて上昇していると見られる。
そこで濃度0と0.5、および0.5と10の間に有意差があるかを統計的な手法で確かめた。

それぞれの濃度に応じた値の平均、分散、標準偏差は以下の通りである。

\begin{table}[H]
\begin{center}
\begin{tabular}{|c|c|c|c|}
\hline
濃度   & 0           & 0.5         & 10          \\ \hline
平均   & 2996.888953 & 4249.949444 & 5366.179027 \\ \hline
分散   & 390635.796  & 1441256.756 & 1993553.633 \\ \hline
標準偏差 & 632.4054461 & 1214.089202 & 1431.40844  \\ \hline
\end{tabular}
\end{center}
\end{table}

これをもとに濃度0と0.5、および0.5と10の間でF検定を行った。

\begin{table}[H]
\begin{center}
\begin{tabular}{|c|c|c|}
\hline
& 0-0.5                & 0.5-10               \\ \hline
F値                   & 3.690 & 1.383 \\ \hline
有意水準                 & 0.05        & 0.05        \\ \hline
P(F$\leqq$f) 片側 & 1.90$\times10^{-5}$ & 0.267 \\ \hline
\end{tabular}
\end{center}
\end{table}

結果より0-0.5と0.5-10で共に等分散であることが確認されたため、等分散の2標本でのt検定を行う。

\begin{table}[H]
\begin{center}
\begin{tabular}{|c|c|c|}
\hline
& 0-0.5                & 0.5-10               \\ \hline
t値                   & -6.10 & -3.87 \\ \hline
有意水準                 & 0.05        & 0.05        \\ \hline
P(T$\leqq$t) 両側 & 3.98$\times10^{-8}$  & 2.61$\times10^{-4}$ \\ \hline
\end{tabular}
\end{center}
\end{table}

結果より、p値が有意水準を下回っているので0-0.5と0.5-10の間で共に有意差があることが確認された。
したがってNF$\kappa$Bの核移行の要因としてTNF$\alpha$の濃度変化を想定することは十分な確率で正しいといえる。


\part*{実験2 アイソトープ実習}

\section*{目的}
細胞抽出液のcyclic AMPと放射線のシグナルとの関係を得る

\section*{方法}
実習書に基づきTAの指示にしたがって実験を行った。

\section*{結果・考察}
\begin{itemize}
\mon{1}
ImageJを用いてスポットされた点をグレースケールで評価し、その値を得た。
総カウントとブランクのカウントはそれぞれ400.65,82.05となった。
よって
\[
\frac{B}{T}(\%) = \frac{x-82.05}{318.60} \times 100
\]
となり、CTを加えた各サンプルの濃度ごとに算出した結合率の平均は以下のようになった。

\begin{table}[H]
\begin{center}
\begin{tabular}{|c|c|c|c|c|c|c|c|c|}
\hline
濃度(μg/ml) & 0.625       & 1.25        & 2.5         & 5           & 10          & 20          & 40          & 80          \\ \hline
結合率(\%) & 76.046 & 66.561 & 55.840 & 36.580 & 21.267 & 15.260 & 10.188 & 6.2863 \\ \hline
\end{tabular}
\end{center}
\end{table}

\mon{2}
測定値とともに近似直線を引いたものが以下のグラフである。
\pict{images/graph2.png}{10}
これより、結合率$y$は標準液の濃度$x$を用いて以下のように表される。
\[
y = -15.58 \ln(x) + 66.482
\]

\mon{3}
これをもとに、CT濃度の異なる検体のcAMP濃度を求め、グラフを作成した。
\begin{table}[H]
\begin{center}
\begin{tabular}{|c|c|c|c|c|c|c|c|}
\hline
CT濃度 & PBS         & 0.2         & 2           & 20          & 200         & 2           & 200         \\ \hline
cAMP濃度 & 0.1256 & 0.2592 & 0.3421 & 0.4763 & 0.4501 & 0.1317 & 0.2012 \\ \hline
\end{tabular}
\end{center}
\end{table}

\pict{images/graph3.png}{10}

実習後に配布されたPDFに添付されているグラフと見比べると、
こちらのグラフは理想的な検量線が引けないようなプロットとなっており、
コレラ毒素を出す細胞に不具合があったことが裏付けられる。

\mon{4}
CT濃度が上昇すると、それに応じてcAMP濃度も上昇すると思われるが、
実習後に配布されたPDFに添付されているグラフによると、
CT濃度がある一定の値を超えるとcAMP濃度が頭打ちになることがわかる。
これは細胞に対して与えるコレラ毒素が飽和してしまっているからと考えられる。
検体11$\sim$14からもわかるように、希釈したCT濃度をもとに戻した値のプロットからは直線状の検量線が引けることがわかる。

\mon{5}
サクシニル化cAMP標準液と結合率の関係は直線となる検量線を引くために片対数を用いた。
対数を用いないと、低濃度の領域に点が密集してしまい線をうまく引くことができない。

\mon{6}
抗体濃度を$a$、[$^{125}I$]サクシニル化cAMP濃度を$b$、サクシニル化cAMP濃度をxとおくと、
放射性のサクシニル化cAMPのうち、抗体と結合したものの割合が結合率となるので、$B/T$は以下のようになる。
\[
B/T = \frac{a \times \frac{b}{b+x}}{b} = \frac{a}{b+x}
\]

\end{itemize}

\section*{課題}
\subsection*{1 RIの選択 RIの核種の物理的特性}
\subsubsection*{1-1 RIが放射する放射線の種類}
同じ元素で中性子の数が違う同位体の中でも不安定なものは時間とともに放射性崩壊して放射線を発する。これを放射性同位体(RI)と言う。
放射性同位体はアルファ崩壊により原子番号と質量数の異なる核種へ、ベータ崩壊により同質量数で原子番号の異なる核種へと放射性崩壊を起こすほか、ガンマ崩壊という質量数も原子番号も変わらない崩壊の種類もある。
\subsubsection*{1-2 放射線のエネルギーと測定機器の選択}
放射線を測定する機器には様々なものがあり目的や状況に応じて使う必要がある。
\begin{itemize}
\item 電離箱式サーベイメータ
ガンマ線によって生じたイオンを中心電極で収集し、その電流を測定することによって線量を評価する。
測定範囲は概ね0.1$\sim$1$\mu$Sv/h以上で使用可能である。正確な線量評価が可能。
\item GM計数管
電離箱よりも高い電圧を中心電極と壁材の間に加え、放射線が通ると放電する。この放電によるパルスを計測することで放射線量を測定する。
電離箱よりも安価で広く普及しているが、エネルギー特性が悪く正確な線量評価が難しいという欠点がある。
\item シンチレーション検出器
放射線によって蛍光(シンチレーション)を発する物質をシンチレーターと呼び、固体のまま、あるいは液体・固体中に溶かし込んで用いられる。
シンチレーション検出器は、放射線のエネルギーを蛍光(シンチレーション)に変換し、さらに光電子増倍管により電気的パルスとして計数する。
エネルギーの異なる放射線を放出する核種が混在している場合でも、チャンネルを最適に設定することで、特定のRIを測定することが可能である。
\end{itemize}
\subsubsection*{1-3 放射線の物理学的半減期}
放射性物質の放射能が半分に減少するまでの時間を放射性物質の「物理学的半減期」という。物理学的半減期は放射性物質に固有の値で、温度、圧力などに依存しない。
例えば、ヨウ素131の物理学的半減期は約8日、セシウム137は約30年である。
また、体内に取り込まれた放射性物質は、「物理学的半減期」に伴い減衰するだけでなく、代謝や排泄などにより体外に排出されることでも減衰する。
このような生物学的な過程によって体内の放射性物質が半分に減少する期間を放射性物質の「生物学的半減期」という。
放射性物質が体内に取り込まれた場合には、「物理学的半減期」に従った減衰と、「生物学的半減期」に従った減少が同時進行する。
両方の半減期を考慮した半減期がを実効半減期という。
\subsection*{2 市販品の標識化合物について}
比放射量と放射能濃度の違いについて。
\begin{itemize}
\mon{2-1}
比放射能または質量放射能は、放射性同位体を含む物質の単位質量あたりの放射能の強さのことである。つまり単位時間・単位質量あたりに同一の放射性物質が壊変する回数のことである。
放射能の単位は、国際単位系ではベクレル(Bq)を用いるが、そのほかにもキュリー(Ci)やラザフォード(Rd)という単位も存在する。
したがって比放射能の次元は、$M^{-1}T^{-1}$であり、単位は、Bq/kg、Bq/g、Ci/gなどである。比放射能が大きい放射性物質ほど、多くの放射線を出す能力があると言える。
\mon{2-2}
水や空気あるいは金属など、物質の単位容積あるいは単位重量等の中に含まれている放射能の量を放射能濃度という。単位は、液体および気体の場合Bq/cm3、固体の場合はBq/gなどを用いる。
\end{itemize}
\subsection*{RIAの原理}
ラジオイムノアッセイは放射免疫分析法とも呼ばれ、微量生体成分の測定法として生物学、医学の分野で広く利用されている。
種々の成分が多量に含まれている生体試料中にng〜pgの極微量存在するホルモンのような特定物質の定量に適している。
RIAは測定しようとする抗原と放射性核種で標識した抗原とが競合的に抗体と結合することを利用し、結合した複合物質の放射能を測定して微量物質を定量する。
ホルモンの他に腫瘍マーカー、特殊蛋白にも適用できる。また、抗原でなく抗体を標識して測定する方法もある。放射性核種としては$^{125}I$が最もよく用いられる。
\subsection*{課題1}
コレラは激しい水様性の下痢を伴う致死的な細菌性腸菅感染症である。この下痢は菌の産生する毒素によって引き起こされる。
コレラ毒素はAサブユニット1分子、Bサブユニット5分子から構成され、Aサブユニットは三量体GTP結合タンパク質であるGsの$\alpha$サブユニットをADP-リボシル化する。
このADP-リボシル化によりアデニル酸シクラーゼが恒常的に活性化され、細胞内cAMP濃度が上昇する。
Bサブユニットは標的細胞表面のGM1ガングリオシドに結合し、細胞内にAサブユニットを送り込む働きがある。
\subsection*{課題2}
Aタンパクは翻訳後、菌自ら分泌するタンパク分解酵素や腸管内のトリプシンによってニックが入り、SS結合で結ばれたA1,A2の2つのフラグメントとなる。
Bオリゴマーは、細胞表面のレセプター、GM1ガングリオシドと強く結合し、 コレラ毒素が細胞表面に吸着する役割を担っている。
レセプターに結合後、 A1-A2間のSS結合はグルタチオン等の還元物質により切断され、A1サブユニットは細胞内に押し込まれる。
細胞内に押し込まれたA1フラグメントは、NADを認識し、そのADPリボ-ス基を切断し、アデニールシクラーゼの活性を制御しているタンパクGs$\alpha$成分にADPリボシル基を転移する。
このADPリボシル化により、GS$\alpha$はGTPase活性を失い、アデニル酸シクラーゼを不可逆的にさせる。
制御タンパクは、GTP型とGDP型をとり、GS$\alpha$の不活性化によってGTP型となり、 アデニル酸シクラーゼの持続的な活性化のためにcAMPの細胞内濃度が高まる。
cAMP上昇から下痢までのメカニズムは明らかになっていないが、 cAMPの上昇によってタンパクカイネースの活性化へと導かれ、
いくつかのタンパクリン酸化の過程を経た後、イオン輸送に関するタンパクのリン酸化に起因して、腸管上皮細胞にCL-が蓄積し、管腔側に対する膜透過性も昂進し、
主にクリプト細胞から腸管腔へCL-が分泌され、また、絨毛先端部細胞からのNa+の能動的吸収の抑制を起こすなどして下痢を起こすと考えられている。

\subsection*{参考資料}
\footnotesize
\begin{itemize}
\item http://www.microbio.med.saga-u.ac.jp/Lecture/kohashi-sb1/part1/
\item http://radiation.shotada.com/chapter/03/
\item https://www.jaea.go.jp/
\item http://www.jsmp.org/wp-content/uploads/2013/03/m-radiation.pdf
\item https://www.mhlw.go.jp/stf/shingi/2r9852000002wduw-att/2r9852000002we6w.pdf
\end{itemize}

\end{document}

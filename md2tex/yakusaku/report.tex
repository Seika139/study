\documentclass[a4paper,papersize,dvipdfmx]{jsarticle}
\usepackage{ascmac}
\usepackage{mathtools, amssymb,bm}
\usepackage{comment}
\usepackage[hiresbb]{graphicx}
\usepackage{tcolorbox,color}
\usepackage{here}
\usepackage{multirow} % 表でセルを結合しているときに必要
\usepackage{listings} %日本語のコメントアウトをする場合jlistingが必要
\usepackage{tabularx} % 表の列幅を等しくする


%ここからソースコードの表示に関する設定
% 参考 https://qiita.com/ta_b0_/items/2619d5927492edbb5b03
\lstset{
  basicstyle={\ttfamily},
  identifierstyle={\small},
  commentstyle={\smallitshape},
  keywordstyle={\small\bfseries},
  ndkeywordstyle={\small},
  stringstyle={\small\ttfamily},
  frame={tb},
  breaklines=true,
  columns=[l]{fullflexible},
  numbers=left,
  xrightmargin=0zw,
  xleftmargin=3zw,
  numberstyle={\scriptsize},
  stepnumber=1,
  numbersep=1zw,
  lineskip=-0.5ex
}

\newcolumntype{Y}{&gt;{\centering\arraybackslash}X} %中央揃え

\tcbuselibrary{raster,skins,breakable}

\newcommand{\pic}[1]{\begin{center} \includegraphics[width=1.0\linewidth,clip]{#1} \end{center}}   %写真用
\newcommand{\pict}[2]{\begin{center} \includegraphics[width= {#2} cm]{#1} \end{center}}   %写真用
\newcommand{\piccap}[3]{\begin{figure}[H] \centering \includegraphics[width= {#2} cm]{#1} \caption{#3} \label{fig {#1}} \end{figure}} %キャプションつき画像
\newcommand{\redunderline}[1]{\textcolor{red}{\underline{¥textcolor{black}{#1}}}}   %赤いアンダーライン
\newcommand{\mon}[1]{\item[({#1})] \ }
\newcommand{\ctext}[1]{\raise0.2ex\hbox{\textcircled{\scriptsize{#1}}}}%文字を丸囲みする(2桁の数字までならいける)

% ソースコードを入れる方法
%\begin{lstlisting}[caption=hoge,label=fuga]
% #include<stdio.h>
% int main(){
%    printf("Hello world!");
% }
% \end{lstlisting}


% 複数図を横に並べるときのヒント http://wright.mydns.jp/?p=704
% \begin{figure}{H}
% \centering
% \begin{minipage}{0.22\hsize}
% \piccap{}{}{}
% \end{minipage}
% \begin{minipage}{0.06\hsize}
% \hspace{2mm}
% \end{minipage}
% \begin{minipage}{0.22\hsize}
% \piccap{}{}{}
% \end{minipage}
% \end{figure}

% 画像を貼る時はjpgかjpegで、pngはうまくいかない時もある

%\itemを四角で囲った数字にする場合は以下のコメントアウトを消す
%\renewcommand{\labelenumi}{\textbf{\framebox[1.5zw]{\theenumi}}}


%enumerateの2階層めのカウンタを1,2,3, にする時は以下のコメントアウトを消す
\renewcommand{\theenumii}{\arabic{enumii}}

%enumerateのカウンタについては以下を参照
% http://www3.otani.ac.jp/fkdsemi/pLaTeX_manual/kajyo.html


%enumerateの番号の出力形式を変更するには、カウンタの値を出力する命令を定義し直す。
%レベル	カウンタ	出力する命令	デフォルトの出力
%1	enumi	¥theenumi	アラビア数字(1,2,3,・・・)
%2	enumii	¥theenumii	小文字のアルファベット(a,b,c,・・・)
%3	enumiii	¥theenumiii	小文字のローマ数字(小文字のローマ数字(ⅰ,ⅱ,ⅲ,・・・)
%4	enumiv	¥theenumiv	大文字のアルファベット(A,B,C,・・・)
%例:¥enumiカウンタを大文字のローマ数字で出力する設定
% ¥renewcommand{¥theenumi}{¥Roman{enumi}}

% 番号の出力形式
%命令	出力形式
%¥arabic	アラビア数字(1、2、3、・・・)
%¥roman	ローマ数字(ⅰ、ⅱ、ⅲ、・・・)
%¥Roman	ローマ数字(Ⅰ、Ⅱ、Ⅲ、・・・)
%¥alph	アルファベット(a、b、c、・・・)
%¥Alph	アルファベット(A、B、C、・・・)

% ページ番号を消す場合は以下のコメントアウトを消す
%\pagestyle{empty}

\begin{document}

\title{薬学実習5 薬品作用学教室}
\author{10191043 8班 鈴木健一}
%作成日を入れる場合は消す
\date{}
\maketitle

%以下の3つからフォントサイズを選択するとよい
%\footnotesize
%\small
%\normalsize

\part*{実験1 動物の行動に及ぼす薬物の作用}

\section*{1-1 記憶・学習行動に対する薬物の効果}

\subsection*{概要}
マウスが電気刺激による痛みを記憶する過程において、記憶の「獲得」「固定」「再生」のそれぞれのフェーズにおいて
薬物を投与して、薬物がどの記憶段階に影響を及ぼすかを測定する。

\subsection*{方法}
実習書に則って行った。

\subsection*{結果}
我々の班での実験結果は以下のようになった。

\begin{table}[H]
\centering
\begin{tabular}{|c|c|c|c|c|c|c|}
\hline
& マウス                  & 体重(g) & 薬物   & 投与量(ml)     & 暗室時間day1(秒) & 暗室時間day2(秒)       \\ \hline
\multirow{2}{*}{実験1} & 1     & 31.3 & 生理用食塩水      & 0.313       & 5           & 300以上 \\ \cline{2-7}
& 2     & 30.2 & scopolamine & 0.302       & 102         & 227   \\ \hline
\multirow{2}{*}{実験2} & 3     & 28.2 & 生理用食塩水      & 0.282       & 8           & 300以上 \\ \cline{2-7}
& 4     & 30.4 & scopolamine & 0.304       & 5           & 300以上 \\ \hline
\multirow{2}{*}{実験3} & 5     & 30.9 & 生理用食塩水      & 0.309       & 8           & 300以上 \\ \cline{2-7}
& 6     & 31.3 & scopolamine & 0.313       & 6           & 300以上 \\ \hline
\end{tabular}
\end{table}

また、実験を行った全ての班の結果からそれぞれのマウスの群が暗室に入るまでの時間の中央値と暗室に入るまでに300秒以上かかったマウスの個体数をまとめた表が以下である。

\begin{table}[H]
\centering
\begin{tabular}{|c|c|c|c|c|c|c|}
\hline
\multicolumn{2}{|c|}{\multirow{2}{*}{}} & \multirow{2}{*}{薬物} & \multicolumn{2}{c|}{中央値} & \multicolumn{2}{c|}{300秒以上の個体数} \\ \cline{4-7}
\multicolumn{2}{|c|}{}                  &                     & day1(秒)     & day2(秒)    & day1(匹)        & day2(匹)        \\ \hline
\multirow{2}{*}{実験1}         & 1        & 生理用食塩水              & 10          & 300        & 0              & 14             \\ \cline{2-7}
& 2        & scopolamine         & 27.5        & 143        & 0              & 6              \\ \hline
\multirow{2}{*}{実験2}         & 3        & 生理用食塩水              & 12.5        & 300        & 0              & 14             \\ \cline{2-7}
& 4        & scopolamine         & 11.5        & 300        & 0              & 14             \\ \hline
\multirow{2}{*}{実験3}         & 5        & 生理用食塩水              & 15          & 300        & 0              & 14             \\ \cline{2-7}
& 6        & scopolamine         & 11          & 300        & 0              & 14             \\ \hline
\end{tabular}
\end{table}

\subsection*{考察}
中央値および暗室に入るまでに300秒以上かかった個体数のデータから分かるように
scopolamine投与群と対照群で有意な差が現れたのは実験1の記憶の獲得における過程である。
他の段階でscopolamineを投与しても対照群と同じように暗室と電気刺激を関連する記憶が定着していたが、
記憶の獲得における段階でscopolamineを投与されると暗室と電気刺激を関連する記憶が定着しないマウスが多いことが分かる。
このことからscopolamineには記憶の獲得に影響を及ぼし、記憶の定着を阻む効果があることが想定される。







\section*{1-2 催眠に対する薬物の作用}

\subsection*{概要}
4匹のマウスにそれぞれ生理用食塩水、chlorpromazine、diazepam、caffeineを投与し、
20分後にhexobarbitalを与える。このときの催眠に至る行動から立ち上がるまでの行動を観察する。
同時に正向反射消失時間の測定を行う。

\subsection*{方法}
実習書に則って行った。

\subsection*{結果}


\begin{table}[H]
\centering
\begin{tabular}{|c|c|c|c|c|c|c|}
\hline
マウス & 体重(g) & 投与した薬物             & 投与量(ml) & 正向反射消失時間(min) & 薬品投与時刻 & 催眠薬投与時刻 \\ \hline
1 & 29.6  & 生理食塩水(ml)          & 0.296   & 0             & 15:23  & 15:43   \\ \hline
2 & 29.9  & chlorpromazine HC1 & 0.299   & 0             & 15:22  & 15:42   \\ \hline
3 & 29.8  & diazepam           & 0.298   & 87            & 15:26  & 15:47   \\ \hline
4 & 30.7  & caffeine           & 0.307   & 0             & 15:23  & 15:43   \\ \hline
\end{tabular}
\end{table}

\subsubsection*{マウス1の様子}
16:19 音に反応しなくなる

\subsubsection*{マウス2の様子}
15:40 起きてはいるが、ほとんど動かなくなる

16:13 寝始める

\subsubsection*{マウス3の様子}
15:35 起きてはいるが、動くときに這うような動き 体があまり動いていない

15:40 完全に目を閉じて寝始めた

15:52 正向反射消失

16:05 音に対する反応がなくなる

16:40 寝ていたが、背中の部分の痙攣が激しくなる

\subsubsection*{マウス4の様子}
常に落ち着きなく動き回る

15:54 激しく暴れる。尾がぴんと伸びている

15:56 歩こうとしているが後ろ足をうまく使えていない

16:24 あまり動かなくなったが起きている

\

※16:35にフラッシュをたいたが、1$\sim$4いずれのマウスにおいても光に対しては反応を示さなかった。


\subsection*{考察}
diazepamを投与したマウスは眠りが深く、27分間正向反射消失の状態が続いた。
観測終了後も足元が覚束なくてとても眠そうな様子だった。
一方caffeineを投与したマウスは大変活発で他のマウスが眠りにつこうとすると、暴れ回って起こすような動作を見せた。

以上よりdiazepamには強い催眠作用があり、逆にcaffeineには覚醒を促す作用があると考えられる。




\section*{1-3 痙攣試験}

\subsection*{概要}
3匹のマウスにそれぞれ生理用食塩水、picrotoxin,tremorineを投与し、
痙攣の発現および外部刺激に対する反応を観察する。

\subsection*{方法}
実習書に則って行った。

\subsection*{結果}
\begin{table}[H]
\centering
\begin{tabular}{|c|c|c|c|}
\hline
マウス & 体重(g) & 薬物         & 投与量(ml) \\ \hline
1 & 29.2  & 生理食塩水      & 0.292   \\ \hline
2 & 28.7  & picrotoxin & 0.287   \\ \hline
3 & 29.5  & tremorine  & 0.295   \\ \hline
\end{tabular}
\end{table}


\subsubsection*{マウス1の様子}
何も投与していないマウスと同じように自由気ままな様子であった。

\subsubsection*{マウス2の様子}
\begin{itemize}
\mon{14:40}	ぐったりと眠そうな目をしている。他のマウスに乗られると場所を変えてまた寝る。胴体を中心に全身に軽い震えが見られる。
\mon{14:43}	尻尾をピンと伸ばしもがき始める。他のマウスに乗られても無反応。
\mon{14:43}	15秒ほど痙攣する。
\mon{14:45}	目を閉じ静止する尻尾も脱力している。
\mon{14:47} 目を開けて少し動く。鼻の毛も動いている。動きは前より少ないが復活したようである。
\mon{14:51}	激しく痙攣する。尻尾がワイパーのように左右に動いてのたうちまわる。
\mon{14:51}	横向きに倒れ震えている。手足が硬直し、その後寝たまま走るような動きを見せる。その後震えつつ起き上がる。静止して左右に震えている。
\mon{14:54}	尾が反り上がり、網目に頭を突っ込み鳴く。少し顔を掻くような仕草を見せたのち両手を揃えて激しく震える。30秒ほどで軽減し、静止して左右に軽く震えている。
\mon{14:55}	再び強く震え出す。尾は脱力しているが根元はピンと張っている。
\mon{14:57}	留まって少し震えている。
\mon{14:59}	ケージを持ち上げて移動させている最中に激しく暴れ出す。10秒も経たずに昏睡状態になる。
\mon{15:02}	その後目が開き、横向きに寝たまま震えた後停止する。
\mon{15:03}	完全に無反応になる。耳や尾が白く血流がない。
\mon{15:07}	目の色が暗い。
\mon{15:18}	目が完全に黒く透明なになった。死亡したようだ。
\end{itemize}

\subsubsection*{マウス3の様子}
\begin{itemize}
\mon{14:43}	おとなしくなり始める。
\mon{14:54}	ほとんど歩かずとどまっている。
\mon{14:55}	他のマウスに乗られても顔の向きを変える程度で気にしない。
\mon{14:57}	震えながら少し歩いているが、10秒ほどで治る。尾は脱力しているが目ははっきりしている。
\mon{15:03}	つついてみると身震いをしたが逃げたりはしない。
\mon{15:05}	尾の震えが見られる。前進を細かく震わせつつ歩く。静止している間は震えが少ない。金網を掴めずに足を滑らせたりしている。鼻の毛は忙しなく動いている(1はほとんど動いてない)
\mon{15:07}	ときおり左後ろ足を持ち上げるような動作を見せる。
\mon{15:09}	つついても動かず、ゆったり震えている。
\mon{15:09}	手や足だけが思い出したようにたまに動く。伸びきってはおらず畳んでいる。
\mon{15:10}	尻尾が硬直している。前進すると震えが激しくなる。
\mon{15:10}	止まっている間の震えも激しくなった。唾液の分泌が見られる。動き出そうとすると震えが激しくなるらしい?実際に前進しなくてもそぶりを見せると震える
\mon{15:10}	顔を掻くような仕草をするがうまく手が使えてないようだ。
\mon{15:10}	震える時は手足を縮こまらせている。寒そう。
\mon{15:14}	憔悴した様子で唾液がすごい分泌されており泡になっている。
\mon{15:17}	突然歩き出すがやはり歩き出すと震える。鼻の毛の動きは少なくなっている。涙も出ており眼表面に液滴が溜まっている。
\mon{15:23}	動作時の震えが激しい。左右に震えている。また他のケージのマウスに尻尾咬まれていた。尻尾が硬直して震えているが胴は震えてない。排便してるから震えているという見方もある。
\mon{15:26}	両手足をこまめに動かしているが腹を地面につけたままである。尻尾は脱力と硬直を5秒単位くらいで繰り返している。
\mon{15:30}	尾は脱力しているが耳はピンと立っている。両腕をしっかり畳んでいる。唾液の分泌はあるもののさっきよりは控えめである。
\mon{15:34}	かおをピクつかせている。口は唾液でびちゃびちゃしている。
\mon{15:37}	静止しているが左右に大きく震えている。
\mon{15:47}	手を網の中に突っ込んだまま唾液を垂れ流し3秒ほど激しく身震いする。その後も震えながら歩行するなどの今まで見たような動きを繰り返している。
\mon{15:47}	やはり歩行以外で突然体を震わせているのは排泄の時のようにも見える・
\mon{15:53}	たまに手足を動かすが動きは大幅に減っている。口元には泡が見られる。
\mon{15:57}	スマホの懐中電灯の光を当てると寝てたマウス1は起きてチョロチョロ動いたのに対し3はつついても無反応である。
\mon{15:57}	その後も時折動くが後はフンを垂れ流したままボーッとしてる。生きてはいる。
\end{itemize}

\subsection*{考察}
picrotoxinを投与したマウスは30分以上痙攣や震える様子を見せたが、最終的には死に至った。
tremorineを投与したマウスは一時は憔悴した様子を見せ、口元の泡がすごかった。また震えも激しかった。




\part*{実験2 鎮痛剤の作用}


\subsection*{方法}
実習書に則ってtail-pinch法と酢酸writhing法で鎮痛剤の効力を測定した。

\subsection*{結果}

\subsubsection*{tail-pinch法}

我々の班では以下のような結果になった。

\begin{table}[H]
\centering
\begin{tabular}{|c|c|c|c|c|c|c|c|}
\hline
マウス & 体重(g) & 薬品名                                  & 投与量(ml) & 15min & 30min & 60min & 90min \\ \hline
赤1  & 29.0  & \multirow{2}{*}{saline}              & 0.29    & 1.49  & 0.31  & 0.24  & 0.5   \\ \cline{1-2} \cline{4-8}
赤2  & 30.4  &                                      & 0.304   & 0.61  & 0.5   & 0.79  & 0.6   \\ \hline
赤3  & 29.7  & \multirow{2}{*}{morphine}            & 0.297   & 0.7   & 0.88  & 1.43  & 1.62  \\ \cline{1-2} \cline{4-8}
赤4  & 30.6  &                                      & 0.306   & 0.46  & 0.88  & 1     & 0.92  \\ \hline
青1  & 29.6  & \multirow{2}{*}{naloxone + morphine} & 0.296   & 0.27  & 0.48  & 0.79  & 0.92  \\ \cline{1-2} \cline{4-8}
青2  & 28.5  &                                      & 0.285   & 0.46  & 1.5   & 0.85  & 0.86  \\ \hline
青3  & 30.8  & \multirow{2}{*}{ibuprofen}           & 0.308   & 0.31  & 0.43  & 0.35  & 0.72  \\ \cline{1-2} \cline{4-8}
青4  & 31.5  &                                      & 0.315   & 0.93  & 0.49  & 0.83  & 0.52  \\ \hline
\end{tabular}
\end{table}

また、全体の結果を用いてそれぞれの薬物が生理用食塩水に比べて有意な鎮痛効果があるかを確認するためにカイ二乗検定を行った。
以下の表はtail-pinch法において2秒以上反応しなかったマウスの個体数をまとめたものである。
当初は6秒以上反応しなかった個体数をまとめる予定であったが、
6秒以上反応しなかったマウスの個体数が非常に少なかったので鎮痛効果があったとする時間の区切りを2秒とした。
マウスの全体の個体数は30である。

\begin{table}[H]
\centering
\begin{tabular}{|c|c|c|c|c|}
\hline
時間    & saline & morphine & naloxone+morphine & ibuprofen \\ \hline
15min & 3      & 12       & 2                 & 2         \\ \hline
30min & 3      & 16       & 1                 & 3         \\ \hline
60min & 2      & 15       & 0                 & 5         \\ \hline
90min & 2      & 14       & 4                 & 3         \\ \hline
\end{tabular}
\end{table}

このデータをもとにscipyでカイ二乗検定を行った。そのときに用いたソースコードを以下に添付する。

\begin{lstlisting}[caption=chi\_sq.py]
#*{!/usr/bin/env python}
#*{-*- coding: utf-8 -*-}

import numpy as np
import pandas as pd
import scipy as sp
from scipy import stats
import os

result_path = os.path.join(os.path.dirname(__file__),'result.txt')
raw_data = [[3,12,2,2],[3,16,1,3],[2,15,0,5],[2,14,4,3]]
index = ["15min","30min","60min","90min"]
columns = ["saline","morphine","naloxone","ibuprofen"]
master = pd.DataFrame(raw_data,index=index,columns=columns)

with open(result_path,'w') as f:
  f.write("")

for idx,row in master.iterrows():
  text = f'{idx},,\n'
  for i in range(4):
    for j in range(i):
      si = np.array([row[i],row[j]])
      si_bar = 30 - si
      cl = [columns[i],columns[j]]
      df = pd.DataFrame([si,si_bar],columns=cl)
      squared,p,dof,ef = stats.chi2_contingency(df)
      text += f'{columns[i]}-{columns[j]}'
      yui = '有意差あり' if squared > 3.841 else '有意差なし'
      text += f',{squared:.3f},{yui}\n'
  with open(result_path, 'a') as f:
    f.write(text)
\end{lstlisting}

4種類の薬品をそれぞれ比較した時の有意差をカイ二乗値によって判定した。
自由度1のカイ二乗分布に従う確率変数の上側5\%点は3.841であるので、この数値よりカイ二乗値が大きいものを
有意差ありと判定した。

\begin{table}[H]
\centering
\begin{tabular}{|c|r|c|r|c|r|c|r|c|}
\hline
\multirow{2}{*}{薬物} & \multicolumn{2}{c|}{15min}       & \multicolumn{2}{c|}{30min}       & \multicolumn{2}{c|}{60min}       & \multicolumn{2}{c|}{90min}       \\ \cline{2-9}
& \multicolumn{1}{c|}{$\chi^2$} & 有意差 & \multicolumn{1}{c|}{$\chi^2$} & 有意差 & \multicolumn{1}{c|}{$\chi^2$} & 有意差 & \multicolumn{1}{c|}{$\chi^2$} & 有意差 \\ \hline
saline-morphine     & 5.689                      & 有   & 11.091                     & 有   & 11.819                     & 有   & 10.312                     & 有   \\ \hline
saline-naloxone     & 0                          & 無   & 0.268                      & 無   & 0.517                      & 無   & 0.185                      & 無   \\ \hline
saline-ibuprofen    & 0                          & 無   & 0                          & 無   & 0.647                      & 無   & 0                          & 無   \\ \hline
morphine-naloxone   & 7.547                      & 有   & 16.088                     & 有   & 17.422                     & 有   & 6.429                      & 有   \\ \hline
morphine-ibuprofen  & 7.547                      & 有   & 11.091                     & 有   & 6.075                      & 有   & 8.208                      & 有   \\ \hline
naloxone-ibuprofen  & 0                          & 無   & 0.268                      & 無   & 3.491                      & 無   & 0                          & 無   \\ \hline
\end{tabular}
\end{table}




\subsubsection*{酢酸writhing法}

我々の班では以下のような結果になった。

\begin{table}[H]
\centering
\begin{tabular}{|c|c|c|c|c|c|}
\hline
マウス & 体重(g) & 薬品名                        & 投与量(ml) & 酢酸投与量(ml) & writhing responseの回数 \\ \hline
黒1  & 29.4  & \multirow{2}{*}{saline}    & 0.294   & 0.294     & 16                   \\ \cline{1-2} \cline{4-6}
黒2  & 27.7  &                            & 0.277   & 0.277     & 21                   \\ \hline
黒3  & 28.7  & \multirow{2}{*}{morphine}  & 0.287   & 0.287     & 0                    \\ \cline{1-2} \cline{4-6}
黒4  & 27.9  &                            & 0.279   & 0.279     & 1                    \\ \hline
黒5  & 30.6  & \multirow{2}{*}{ibuprofen} & 0.306   & 0.306     & 23                   \\ \cline{1-2} \cline{4-6}
黒6  & 28.1  &                            & 0.281   & 0.281     & 8                    \\ \hline
\end{tabular}
\end{table}

また、全体の結果を用いてそれぞれの薬物が生理用食塩水に比べて有意な鎮痛効果があるかを確認するためにt検定を行った。
以下の表は酢酸writhing法においてマウスがwrithing responseをした回数をまとめたものである。

\newcolumntype{C}{>{\centering\arraybackslash}p{9mm}}
\begin{table}[H]
\centering
\begin{tabular}{|c|C|C|C|C|C|C|}
\hline
班  & \multicolumn{2}{c|}{生理食塩水} & \multicolumn{2}{c|}{morphine} & \multicolumn{2}{c|}{ibuprofen} \\ \hline
1 & 46           & 26          & 0             & 0             & 21             & 1             \\ \hline
2 & 44           & 36          & 0             & 0             & 31             & 28            \\ \hline
3 & 63           & 31          & 3             & 4             & 31             & 29            \\ \hline
4 & 17           & 27          & 1             & 0             & 14             & 19            \\ \hline
5 & 4            & 2           & 0             & 0             & 33             & 12            \\ \hline
6 & 17           & 13          & 0             & 1             & 7              & 18            \\ \hline
7 & 6            & 13          & 0             & 0             & 9              & 3             \\ \hline
8 & 16           & 21          & 0             & 0             & 23             & 8             \\ \hline
9 & 28           & 6           & 0             & 0             & 25             & 30            \\ \hline
10 & 17           & 15          & 0             & 0             & 1              & 14            \\ \hline
11 & 12           & 23          & 0             & 0             & 14             & 17            \\ \hline
12 & 16           & 12          & 0             & 0             & 0              & 10            \\ \hline
13 & 11           & 1           & 0             & 0             & 0              & 0             \\ \hline
14 & 19           & 42          & 0             & 0             & 1              & 8             \\ \hline
15 & 24           & 31          & 0             & 0             & 6              & 5             \\ \hline
\end{tabular}
\end{table}

それぞれの濃度に応じた値の平均、分散、標準偏差は以下の通りである。

\begin{table}[H]
\centering
\begin{tabular}{|c|c|c|c|}
\hline
& saline  & morphine & ibuprofen \\ \hline
平均   & 21.300  & 0.300    & 13.933    \\ \hline
分散   & 197.743 & 0.810    & 114.796   \\ \hline
標準偏差 & 14.062  & 0.900    & 10.714    \\ \hline
\end{tabular}
\end{table}

これをもとに生理用食塩水とmorphineおよびibuprofenとの間でF検定を行った。

\begin{table}[H]
\centering
\begin{tabular}{|c|c|c|}
\hline
& saline-morphine & saline-ibuprofen \\ \hline
F値                  & 244.128         & 1.723            \\ \hline
P(F\textless{}f) 両側 & $1.69 \times 10^{-27}$     & 0.1490         \\ \hline
P(T\textless{}t) 両側 & $7.08 \times 10^{-9}$     & 0.0287         \\ \hline
\end{tabular}
\end{table}

saline-morphineではF検定による結果で両側確率が5\%を下回ったため不等分散でのt検定を行い、
saline-ibuprofenでは両側確率が5\%を上回ったため等分散でのt検定を行った。
その結果、両者ともにt検定によって得られたp値の両側確率が有意水準である5\%を下回ったので帰無仮説が棄却された。
よってmorphineとibuprofenはともに生理用食塩水に比べて有意な鎮痛効果があると言える。


\subsection*{考察}

tail-pinch法の結果からは下記の内容が考察として結論づけられる。
\begin{itemize}
\item 全ての時間においてmorphineの投与は他の薬物に比べて有意な鎮痛作用を示す。
\item ibuprofenは生理用食塩水を投与した場合と鎮痛作用が変わらない。つまり鎮痛作用がない。
\item morphineとnaloxoneを両方投与すると鎮痛作用がなくなることからnaloxoneにはmorphineの鎮痛作用を阻害する効果がある。
\end{itemize}

酢酸writhing法の結果からは下記の内容が考察として得られた。
\begin{itemize}
\item morphineとibuprofenとはともに鎮痛作用がある。
\item morphineはibuprofenに比べても有意な鎮痛作用がある。
\end{itemize}

これらの結果からmorphineは明らかに鎮痛作用があることが認められるが、
ibuprofenに関しては鎮痛作用があるかどうかを判定しがたい。
鎮痛作用の検出感度という点ではtail-pinch法よりも酢酸writhing法が具体的な回数で痛みを感じている度合いを測定できることから
酢酸writhing法の結果の方が信頼できるのではないかと思われる。

最後に鎮痛薬の臨床的な使用について記述する。
がん性疼痛に対するオピオイド鎮痛薬の使用は「WHO方式がん疼痛治療法」を基本とし、
痛みの強さによる鎮痛薬の段階的な使用法を示した「WHO三段階除痛ラダー」によって使用すべき鎮痛剤が定められている。
軽度の痛みには第一段階の非オピオイド鎮痛薬を使用するが、これらの薬剤は副作用と有効限界により基本的に標準投与量を超える増量は行わない。
非オピオイド鎮痛薬で効果が不十分なときには、「軽度から中等度の強さの痛み」に用いるオピオイド鎮痛薬を追加する。
第二段階で痛みの緩和が十分でない場合は、第三段階の薬剤に変更する。
ただし,痛みの強さによっては第二段階を省略してよい。
また、いずれの段階においても非オピオイド鎮痛薬は併用するのが原則である。さらに、第三段階でも必要により鎮痛補助薬の使用を検討する。
患者の生命予後の長短にかかわらず、痛みの程度に応じて躊躇せずに必要な鎮痛薬を選択することが重要である。
WHO三段階除痛ラダーにおいてibuprofenは第一段階に、morphineは第三段階に分類されている。

\subsubsection*{参考資料}
\footnotesize
\begin{itemize}
  \item http://www.jiho.co.jp/Portals/0/ec/product/ebooks/book/46601/46601.pdf
  \item https://www.jspm.ne.jp/guidelines/pain/2010/chapter02/02\_04\_01\_06.php
\end{itemize}

\end{document}

\documentclass[a4paper,papersize,dvipdfmx]{jsarticle}
\usepackage{ascmac}
\usepackage{mathtools, amssymb,bm}
\usepackage{comment}
\usepackage[hiresbb]{graphicx}
\usepackage{tcolorbox,color}
\usepackage{here}
\tcbuselibrary{raster,skins,breakable}

\newcommand{\pic}[1]{\begin{center} \includegraphics[width=1.0\linewidth,clip]{#1} \end{center}}   %写真用
\newcommand{\pict}[2]{\begin{center} \includegraphics[width= {#2} cm]{#1} \end{center}}   %写真用
\newcommand{\redunderline}[1]{\textcolor{red}{\underline{¥textcolor{black}{#1}}}}   %赤いアンダーライン
\newcommand{\mon}[1]{\item[({#1})] \ }
\newcommand{\ctext}[1]{\raise0.2ex\hbox{\textcircled{\scriptsize{#1}}}}%文字を丸囲みする(2桁の数字までならいける)

%\itemを四角で囲った数字にする場合は以下のコメントアウトを消す
%\renewcommand{\labelenumi}{\textbf{\framebox[1.5zw]{\theenumi}}}


%enumerateの2階層めのカウンタを1,2,3, にする時は以下のコメントアウトを消す
\renewcommand{\theenumii}{\arabic{enumii}}

%enumerateのカウンタについては以下を参照
% http://www3.otani.ac.jp/fkdsemi/pLaTeX_manual/kajyo.html


%enumerateの番号の出力形式を変更するには、カウンタの値を出力する命令を定義し直す。
%レベル	カウンタ	出力する命令	デフォルトの出力
%1	enumi	¥theenumi	アラビア数字(1,2,3,・・・)
%2	enumii	¥theenumii	小文字のアルファベット(a,b,c,・・・)
%3	enumiii	¥theenumiii	小文字のローマ数字(小文字のローマ数字(ⅰ,ⅱ,ⅲ,・・・)
%4	enumiv	¥theenumiv	大文字のアルファベット(A,B,C,・・・)
%例:¥enumiカウンタを大文字のローマ数字で出力する設定
% ¥renewcommand{¥theenumi}{¥Roman{enumi}}

% 番号の出力形式
%命令	出力形式
%¥arabic	アラビア数字(1、2、3、・・・)
%¥roman	ローマ数字(ⅰ、ⅱ、ⅲ、・・・)
%¥Roman	ローマ数字(Ⅰ、Ⅱ、Ⅲ、・・・)
%¥alph	アルファベット(a、b、c、・・・)
%¥Alph	アルファベット(A、B、C、・・・)




\begin{document}

\title{ここにタイトルを記入}
\author{SUZUKEN}
%作成日を入れる場合は消す
\date{}
\maketitle

%以下の3つからフォントサイズを選択するとよい
%\footnotesize
%\small
%\normalsize

\tableofcontents

\part{薬の作用}

\section{アゴニストとアンタゴニスト}


\subsection{アゴニスト}
生体内の受容体分子と結合することで神経伝達物質やホルモンと同様の機能を示す薬のことです。  作動薬とほぼ同義で使われます。


\subsection{アンタゴニスト}
生体内の受容体分子と結合することで神経伝達物質やホルモンの機能を妨げるような薬のことです。拮抗薬とほぼ同義で使われます。

\section{薬物の作用するしくみ}

薬物は受容体、酵素、チャネルといったタンパク質に結合することで作用します。

\subsection{受容体}
細胞膜や細胞質などにあるタンパク質です。受容体と親和性の高い基質が結合するとそれをきっかけに、連続的反応が開始されるようになっています。この連続的に続く反応を引き起こしたり、阻害することにより、薬は作用します。

\pict{imgs/sayo1.png}{10}

\subsection{酵素}
生体内における化学反応を触媒するタンパク質です。基質が結合することで、化学反応を阻害したり、促進したりします。薬が結合することにより、酵素の活性を調節することで薬は作用します。

\pict{imgs/sayo2.png}{10}

\subsection{チャネル}
細胞膜などの膜上にあるタンパク質です。イオンや特定の物質の通路として機能しています。薬が結合することで、チャネルの開閉をコントロールすることにより、薬は作用します。

\pict{imgs/sayo3.png}{10}

\section{代表的な薬物受容体、細胞内情報伝達系}

代表的な薬物受容体は2つ
\begin{enumerate}
\item Gタンパク質共役受容体(GPCR  :  G Protein-coupled receptor)
\item イオンチャネル内蔵型受容体


\end{enumerate}
\subsection{Gタンパク質共役受容体(GPCR : G Protein-coupled receptor)}

Gは、グアニンヌクレオチド結合の略です。GPCRは、受容体とGタンパク質が結合した構造をしています。イメージと、実際のタンパク質結晶のリボン図の1例が以下になります。

\pict{imgs/c3-1.png}{10}

共役する代表的なGタンパク質が大きく3つ存在します。

\subsubsection{Gsタンパク質}
sはstimulate(刺激)の略です。Gsタンパク質と共役する受容体に基質が結合すると、Gsタンパク質を介してアデニル酸シクラーゼ(AC:adenylate cyclase)が刺激(活性化)されます。アデニル酸シクラーゼが活性化されると、細胞内サイクリックAMP(cAMP:cyclic AMP)が増加します。細胞内cAMPが増加することで、様々な生理的反応が引き起こされます。「受容体に基質(メッセンジャー)がやってきた」というメッセージを細胞内に伝える役割に注目しcAMPをセカンドメッセンジャーと呼びます。cAMPの構造は以下のようなものです。

\pict{imgs/c3-2.png}{10}

\subsubsection{Giタンパク質}
iはinhibit(抑制)の略です。Giタンパク質と共役する受容体に基質が結合すると
Giタンパク質を介してアデニル酸シクラーゼ(AC:adenylate cyclase)が抑制されます。アデニル酸シクラーゼが抑制されると、細胞内サイクリックAMP(cAMP:cyclic AMP)が減少します。細胞内cAMPが減少することで、様々な生理的反応が引き起こされます。

\subsubsection{Gqタンパク質}
qの由来には決まった説がありません。
Gqタンパク質と共役する受容体に基質が結合するとGqタンパク質を介してホスホリパーゼCが活性化されます。ホスホリパーゼCは、細胞膜を原料として
ジアシルグリセロール(DG)とイノシトール三リン酸(IP3)を生成します。これらがセカンドメッセンジャーとなって、様々な生理的反応が引き起こされます。


\begin{tcolorbox}[colback=white,colbacktitle=black,coltitle=white,title={まとめ}]
\pict{imgs/c3-3.png}{10}
\end{tcolorbox}


\subsection{イオンチャネル内蔵型受容体}

5量体からなる膜貫通タンパク質です。受容体そのものがイオンチャネルを形成しています。


\begin{itemize}
\item 代表的なイオンチャネル内蔵型の受容体と通過するイオン

\begin{table}[H]
\begin{center}
\begin{tabular}{|c|c|}
\hline
受容体                  & 通過するイオン              \\ \hline
$\rm N_N N_M$ 受容体    & $\rm Na^+$ チャネル      \\ \hline
$\rm GABA_A$ グリシン受容体 & $\rm Cl^-$ チャネル      \\ \hline
グルタミン酸受容体 NMDA型      & $\rm Ca^{2+}$ チャネル   \\ \hline
グルタミン酸受容体 non-NMDA型  & $\rm Na^+ K^+$ チャネル  \\ \hline
5-$\rm HT_3$ 受容体     & $\rm Na^+ aK^+$ チャネル \\ \hline
\end{tabular}
\end{center}
\end{table}

特に$\rm Na^+$ チャネルが N 受容体、$\rm Cl^-$ チャネルが GABA 、グリシン受容体であることが重要です。

\end{itemize}
\section{薬効の個人差が生じる要因}
例えば降圧薬を同じ量投与しても、よく降圧作用が見られる人とあまり降圧作用が見られない人がいます。これを薬効の個人差といいます。

薬効の個人差はさまざまな要因から生じます。代表的な要因は年齢、性別、種差、代謝酵素の遺伝的な違い受容体の遺伝的な違いなどです。

\section{代表的な薬物相互作用の機序}


薬物相互作用とは、複数の薬物が投与された時に薬理作用が変化することです。
薬の吸収、代謝、排泄における相互作用が主な薬物相互作用です。

代表的な薬物相互作用を以下に例としてあげます。

吸収における代表的な薬物相互作用はニューキノロン系抗菌薬と、金属イオンの相互作用です。これは、ニューキノロンと金属イオンが不溶性キレート化合物を形成することにより吸収が低下する相互作用です。


代謝における代表的な薬物相互作用はテオフィリンとシメチジンの相互作用です。シメチジンは、全てのCYPという代謝酵素を阻害する薬物です。テオフィリンはCYP1A2という酵素により代謝されます。シメチジンによりテオフィリンの代謝が阻害されることによりテオフィリンの血中濃度が上がるという相互作用です。

排泄における代表的な薬物相互作用はプロベネシドとペニシリンの相互作用です。プロベネシドは、尿細管に薬を排出する分泌作用を抑制する薬です。ペニシリンは腎排泄の抗菌薬です。プロベネシドにより、ペニシリンの分泌が抑制されることで排泄が遅くなりペニシリンの血中濃度が高いまま維持されるという相互作用です。

注意しなければならないのは相互作用はそれ自体、よい悪いということはないということです。

例えばプロベネシドはペニシリンの血中濃度を維持するために開発された薬であるという背景があります。

また、レボドパとカルビドパという2つの薬剤は相互作用を活かした合剤として今もパーキンソン病の治療薬として使われています。(カルビドパによる、末梢におけるレボドパ代謝酵素阻害を利用することで中枢へのレボドパ移行を実現している合剤です。)

\section{薬物の主作用と副作用、毒性との関連、有害事象}

使用した薬物の目的作用が主作用です。それ以外の作用が副作用です。

副作用には、患者にとって有益なものもあります。副作用の中で、患者にとって好ましくない作用を与える性質を毒性と呼びます。

有害事象とは「薬物使用時において現れる患者にとって好ましくない作用」 かつ「薬との因果関係がはっきりしないものまで含めたもの」のことです。よって有害事象は副作用よりもさらに広い範囲の概念となります。ある事例を与えられた時それが副作用なのか、有害事象なのかといった分類ができることが重要です。

\end{document}
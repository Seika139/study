

\section*{3日目}

\subsection*{結果}
\begin{itemize}
\item 濃縮の操作を始める前から容器の底に結晶があったが水を加えると消えた。
\item 氷浴をしても全然結晶が析出しなかったが、スパーテルで激しく内壁をこすると析出した。
\item その後濾過すると白い粉末状の結晶が得られた。

\end{itemize}
\subsubsection*{粗結晶}
重さ : 14.22 g

6.0 mLのエタノールを加えて湯浴で83 ℃にまで加熱すると結晶が完全に溶けたので、そこから冷却して再結晶を行った。

\subsubsection*{再結晶}
重さ : 9.94 g
収率 : 36.5 %
融点 : 88~90 ℃, 91 ℃

\subsection*{考察}
\subsubsection*{粗結晶と再結晶で重さが5 g異なることについて}
再結晶の操作を行うと少なからず溶媒に結晶が溶けてしまい、一定の量を失ってしまうが、それに加えて反応容器内に取りきれない結晶があることも理由の一つであると考えられる。

\subsection*{課題}
\subsubsection*{(2) 収率について}
1日目に秤量したdiethyl metanol 20.08 g をもとに今日の再結晶で得られたdiethyl 2-acetamidomalonate の収率を求める。
\[(9.94 \times 160.17) \div (20.08 \times 217.22) = 0.365\]

\[\therefore
\ 36.5 \%\]

他の班でdiethyl 2-acetamidomalonate 5 gを超えて得られたと聞かなかったので他班に比べてかなり良い収率が得られたと思われる。

\section*{4日目}
diethyl 2-acetamidomalonate に$\rm PhCh_2Cl, EtoNa, EtOH$を加えてdiethyl 2-acetamido -2- benzilmalonate を合成した。

\subsection*{結果}
\begin{itemize}
\item 今回用いたdiethyl 2-acetamidomalonate は6.32 gである。
\item 油温が90℃になってあたりで液の色が黄色く変化した。
\item 濾過するとろ紙にベージュ色のペースト状の粉がついた。
\item 溶液を濃縮すると淡い黄色の個体が得られた。再結晶では水にほとんど溶けなかったためスパーテルで砕いた。

\end{itemize}
\subsubsection*{TLC}
\pic{imgs4/tlc.jpeg}

\subsubsection*{生成物(乾燥不十分)}
\begin{itemize}
\item 融点 : 67 $\sim$ 68℃
\item 重さ : 10.05 g
\item 収率 : 112.4 $\%$
\end{itemize}
\subsubsection*{生成物}
\begin{itemize}
\item 融点 : 80 $\sim$ 81℃
\item 重さ : 8.01 g
\item 収率 : 89.6 $\%$

\end{itemize}
\subsection*{考察}
\begin{itemize}
\item ろ紙についたベージュ色のペースト状の粉は反応式から考察するとNaClであると考えられる。
\item 結晶を濾取したあとはろ紙やデシケーターで十分乾燥する必要がある。乾燥が足りなかったために収率が高く、融点が低くなってしまった。

\end{itemize}
\subsection*{課題}

\subsubsection*{(1) 反応機構}
\pic{imgs4/hk.jpeg}

\subsubsection*{(3) NaOHではなくEtONaを用いる理由}

$\rm OH^-$のような塩基をdiethyl 2-acetamidomalonate に作用させると色がついている3箇所で反応する可能性がある。このとき赤い色がついている炭素と反応すると$\rm OEt$が$\rm OH$と入れ替わってしまい目的の生成物ができくなってしまう。

\pic{imgs4/st1.jpeg}

そこでNaOEtを用いることで赤や青の色がついた炭素で反応しても生成物が元と変わらないので、目的の生成物に必要な緑の炭素と反応させることができる。

\pic{imgs4/st2.jpeg}

他に赤や青の炭素に作用させない手法としては、立体障害が大きい塩基を用いることで求核攻撃をさせない手法がある。


\section*{5日目}
\subsection*{結果}
\begin{itemize}
\item 加熱還流において溶液がはじめは白く濁っていたが115℃になったあたりで透明になった。
\item 撹拌後にpHが5 $\sim$ 6だったのでさらに塩酸を加えてpHを1にした。
\item 氷浴すると溶液が白く濁った。その後黄色いガム状の沈殿が得られた。
\item 内壁をスパーテルで擦るとガム状のものが白い粉に変わった。再結晶後に得られた結晶は透明度がある白い粉だった。

\end{itemize}
\subsubsection*{TLC}

\pict{imgs5/tlc.jpeg}{10}

\subsubsection*{再結晶}
\begin{itemize}
\item 重さ : 1.96 g
\item 収率 : 48.4%
\item 融点 : 132℃, 134℃


\end{itemize}
\subsection*{考察}

\subsection*{課題}
\pict{imgs5/hk.jpeg}{10}


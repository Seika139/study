


\section*{6日目}

\subsection*{結果}
\begin{itemize}
\item 油温が110℃を超えたあたりで白く濁っていた溶液が透明になった。
\item 反応溶液を減圧留去すると白色の粉が得られた。
\item 水を加えると黄色く濁った溶液となったが、白い粉が完全には溶けきらなかった。

\end{itemize}
\subsubsection*{TLC}
\pict{imgs6/tlc.jpeg}{8}

\subsubsection*{結晶(DL-phenylalanine)}
\begin{itemize}
\item 重さ : 0.18 g
\item 収率 : 22.4 $\%$
\item dec pt : 250℃, 251℃

\end{itemize}
\subsection*{考察}
\subsubsection*{エタノールを用いて再結晶した理由}
フェニルアラニンが双性イオンであるために水によく溶ける。そこで、水と混合可能でかつアミノ酸をとかしにくいエタノールで再結晶することで双性イオンのアミノ酸を得ることができる。

\subsubsection*{分解点(decomposition point)について}
今回用いたフェニルアラニンは双性イオンであるためにイオン結合が非常に強い。そのため融点が非常に高くなる。今回の融点測定管を用いた測定では250℃付近でフェニルアラニンが黄色の液状になったが、これはフェニルアラニンが分解し別の化合物になったからである。つまりフェニルアラニンは加熱するだけでは融点に達するより前に別の化合物になってしまうのである。分解点は融点と異なり化合物の構造が変化するため、冷ましても元の結晶に戻ることはない。

\subsection*{課題}

\subsubsection*{(1) 反応機構}
\pict{imgs6/hk.jpeg}{14}


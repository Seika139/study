

\section*{2日目}
diethyl isonitrosomalonate に亜鉛と酢酸、無水酢酸を加えて diethyl 2-acetamidometanole を合成した。

\subsection*{結果}
三頚フラスコ内でdiethyl isonitrosomalonate に酢酸と無水酢酸を加えた後、亜鉛30 gを10等分して加えていくと溶液が亜鉛で灰青色に染まり、溶液の温度が67度まで上昇した。その後も亜鉛を加える度に溶液の温度が上昇し反応が起こっていることが確認された。

\subsubsection*{TLC}
左の図が最後に亜鉛を加えた時のもの、右が20分後のものである。
また、スポットは右から順に「出発物質、両方、生成物」である。
(img)

\subsection*{考察}
TLCの結果からもわかるように最後に亜鉛を入れた時点で反応物であるdiethyl isonitrosomalonate はなくなっており、代わりに生成物のdiethyl 2-acetamidometanole ができていた。また、今回亜鉛を10回に分けて加えたのは酢酸と反応してZn(CH₃COO)₂ができて亜鉛が失われることを考慮して常に純粋な亜鉛を溶液内に入れておくためだと考えられる。

\subsubsection*{亜鉛の性質}
亜鉛は両性金属であるため、酸だけでなく水やアルカリとも反応して水素を発生する。
\[\rm Zn + 2 H^+ \longrightarrow Zn^{2+} + H_2\]

\[\rm Zn + 2OH^-+ 2H_2O \longrightarrow [Zn(OH)_4]^{2-} + H_2\]

また、空気中の水分などによって自然発火する恐れがあるので扱いには充分注意する必要がある。
\[ \rm 2 Zn + O_2 \longrightarrow 2ZnO\]

\subsection*{課題}

(1) 反応機構


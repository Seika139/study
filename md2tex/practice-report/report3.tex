\documentclass[a4paper,papersize,dvipdfmx]{jsarticle}
\usepackage{ascmac}
\usepackage{mathtools, amssymb,bm}
\usepackage{comment}
\usepackage[hiresbb]{graphicx}
\usepackage{tcolorbox,color}
\usepackage{here}
\tcbuselibrary{raster,skins,breakable}

\newcommand{\pic}[1]{\begin{center} \includegraphics[width=1.0\linewidth,clip]{#1} \end{center}}   %写真用
\newcommand{\pict}[2]{\begin{center} \includegraphics[width= {#2} cm]{#1} \end{center}}   %写真用
\newcommand{\redunderline}[1]{\textcolor{red}{\underline{¥textcolor{black}{#1}}}}   %赤いアンダーライン
\newcommand{\mon}[1]{\item[({#1})] \ }
\newcommand{\ctext}[1]{\raise0.2ex\hbox{\textcircled{\scriptsize{#1}}}}%文字を丸囲みする(2桁の数字までならいける)

%\itemを四角で囲った数字にする場合は以下のコメントアウトを消す
%\renewcommand{\labelenumi}{\textbf{\framebox[1.5zw]{\theenumi}}}


%enumerateの2階層めのカウンタを1,2,3, にする時は以下のコメントアウトを消す
\renewcommand{\theenumii}{\arabic{enumii}}

%enumerateのカウンタについては以下を参照
% http://www3.otani.ac.jp/fkdsemi/pLaTeX_manual/kajyo.html


%enumerateの番号の出力形式を変更するには、カウンタの値を出力する命令を定義し直す。
%レベル	カウンタ	出力する命令	デフォルトの出力
%1	enumi	¥theenumi	アラビア数字(1,2,3,・・・)
%2	enumii	¥theenumii	小文字のアルファベット(a,b,c,・・・)
%3	enumiii	¥theenumiii	小文字のローマ数字(小文字のローマ数字(\UTF{2170},\UTF{2171},\UTF{2172},・・・)
%4	enumiv	¥theenumiv	大文字のアルファベット(A,B,C,・・・)
%例:¥enumiカウンタを大文字のローマ数字で出力する設定
% ¥renewcommand{¥theenumi}{¥Roman{enumi}}

% 番号の出力形式
%命令	出力形式
%¥arabic	アラビア数字(1、2、3、・・・)
%¥roman	ローマ数字(\UTF{2170}、\UTF{2171}、\UTF{2172}、・・・)
%¥Roman	ローマ数字(\UTF{2160}、\UTF{2161}、\UTF{2162}、・・・)
%¥alph	アルファベット(a、b、c、・・・)
%¥Alph	アルファベット(A、B、C、・・・)




\begin{document}

\title{天然物合成化学教室 実習レポート}
\author{10191043 鈴木健一}
%作成日を入れる場合は消す
\date{}
\maketitle

%以下の3つからフォントサイズを選択するとよい
%\footnotesize
%\small
%\normalsize

\begin{flushright}
実験日 : 2019/5/9 $\sim$ 2019/5/16

共同実験者 : 白井孝平
\end{flushright}

\section*{1日目}
diethyl malonate に$\rm{NaNO_2, AcOH, H_2O}$を加えて、diethyl isonitrosomalonate を合成する反応を行った。


\subsection*{結果}
$\rm NaNO_2$ 25 gを10等分してフラスコに入れていくと、始めは透明だった液体が白く濁り始め、次第に黄色く変化していった。また、3つ口フラスコのスリの付近にも黄色い付着物が観測された。最後に分液ロートに移すと黄色い油状の上層と、白濁した下層に分離したことが確認された。

\subsubsection*{TLC}

\pict{imgs1/tlc.jpeg}{6}

\subsection*{考察}

フラスコのスリが黄色く変色したのは$\rm NaNO_2$由来の亜硝酸が以下の反応式に示すような不均化反応によって二酸化窒素に変化し、ガラスやその内部に付着する有機化合物に沈着したからであると考えられる。

\[\rm 3HNO_3 \rightarrow N{O_3}^- + 2NO + H_3O^+\]
\[\rm 2NO + O_2 \rightarrow 2NO_2\]

今回の反応において$\rm NaNO_2$を10分割して加えたり、diethyl malonate に対して三等量ほど用意されていたのは、この不均化反応によって失われる亜硝酸とそれに付随する$\rm NO^+$の不足を補うためである。

\subsection*{課題}
\subsubsection*{(1) 反応機構}

\begin{enumerate}
\item $\rm NO^+$を生じさせる
\pict{imgs1/hk1.jpeg}{10}
\item diethyl malonate がエノール形に互変異性化する
\pict{imgs1/hk2.jpeg}{12}
\item エノール形に $\rm NO^+$が反応して diethyl isonitrosomalonate が生じる
\pict{imgs1/hk3.jpeg}{12}

また、C=C結合よりC=O結合NO方が強いことから、一般的にはエノール形よりケト形の方が安定であるが、今回用いた diethyl malonateのような$\beta$-ジカルボニル化合物は図のように分子内水素結合と共役によるπ電子の非局在化によって安定するので、エノール形の方が割合が高くなることもある。
\pict{imgs1/bck.jpeg}{7}
\end{enumerate}


\section*{2日目}
diethyl isonitrosomalonate に亜鉛と酢酸、無水酢酸を加えて diethyl 2-acetamidometanole を合成した。

\subsection*{結果}
三頚フラスコ内でdiethyl isonitrosomalonate に酢酸と無水酢酸を加えた後、亜鉛30 gを10等分して加えていくと溶液が亜鉛で灰青色に染まり、溶液の温度が67度まで上昇した。その後も亜鉛を加える度に溶液の温度が上昇し反応が起こっていることが確認された。

\subsubsection*{TLC}
左の図が最後に亜鉛を加えた時のもの、右が20分後のものである。
また、スポットは右から順に「出発物質、両方、生成物」である。
\pict{imgs2/tlc.jpeg}{7.5}

\subsection*{考察}
TLCの結果からもわかるように最後に亜鉛を入れた時点で反応物であるdiethyl isonitrosomalonate はなくなっており、代わりに生成物のdiethyl 2-acetamidometanole ができていた。
また、今回亜鉛を10回に分けて加えたのは、酢酸と反応することで$\rm Zn(CH_3COO)_2$ができて亜鉛が失われることを考慮し、常に純粋な亜鉛を溶液内に入れておくためだと考えられる。

\subsubsection*{亜鉛の性質}
亜鉛は両性金属であるため、酸だけでなく水やアルカリとも反応して水素を発生する。
\[\rm Zn + 2 H^+ \longrightarrow Zn^{2+} + H_2\]

\[\rm Zn + 2OH^-+ 2H_2O \longrightarrow [Zn(OH)_4]^{2-} + H_2\]

また、空気中の水分などによって自然発火する恐れがあるので扱いには充分注意する必要がある。
\[ \rm 2 Zn + O_2 \longrightarrow 2ZnO\]

\subsection*{課題}

\subsubsection*{(1) 反応機構}
\pic{imgs2/hk.jpg}



\section*{3日目}

\subsection*{結果}
\begin{itemize}
\item 濃縮の操作を始める前から容器の底に結晶があったが水を加えると消えた。
\item 氷浴をしても全然結晶が析出しなかったが、スパーテルで激しく内壁をこすると析出した。
\item その後濾過すると白い粉末状の結晶が得られた。

\end{itemize}
\subsubsection*{粗結晶}
重さ : 14.22 g

6.0 mLのエタノールを加えて湯浴で83 ℃にまで加熱すると結晶が完全に溶けたので、そこから冷却して再結晶を行った。

\subsubsection*{再結晶}
重さ : 9.94 g
収率 : 36.5 \%
融点 : 88$\sim$90 ℃, 91 ℃

\subsection*{考察}
\subsubsection*{粗結晶と再結晶で重さが5 g異なることについて}
再結晶の操作を行うと少なからず溶媒に結晶が溶けてしまい、一定の量を失ってしまうが、それに加えて反応容器内に取りきれない結晶があることも理由の一つであると考えられる。

\subsection*{課題}
\subsubsection*{(2) 収率について}
1日目に秤量したdiethyl metanol 20.08 g をもとに今日の再結晶で得られたdiethyl 2-acetamidomalonate の収率を求める。
\[(9.94 \times 160.17) \div (20.08 \times 217.22) = 0.365\]

\[\therefore
\ 36.5 \%\]

他の班でdiethyl 2-acetamidomalonate 5 gを超えて得られたと聞かなかったので他班に比べてかなり良い収率が得られたと思われる。

\section*{4日目}
diethyl 2-acetamidomalonate に$\rm PhCh_2Cl, EtoNa, EtOH$を加えてdiethyl 2-acetamido -2- benzilmalonate を合成した。

\subsection*{結果}
\begin{itemize}
\item 今回用いたdiethyl 2-acetamidomalonate は6.32 gである。
\item 油温が90℃になってあたりで液の色が黄色く変化した。
\item 濾過するとろ紙にベージュ色のペースト状の粉がついた。
\item 溶液を濃縮すると淡い黄色の個体が得られた。再結晶では水にほとんど溶けなかったためスパーテルで砕いた。

\end{itemize}
\subsubsection*{TLC}
\pict{imgs4/tlc.jpeg}{7}

\subsubsection*{生成物(乾燥不十分)}
\begin{itemize}
\item 融点 : 67 $\sim$ 68℃
\item 重さ : 10.05 g
\item 収率 : 112.4 $\%$
\end{itemize}
\subsubsection*{生成物}
\begin{itemize}
\item 融点 : 80 $\sim$ 81℃
\item 重さ : 8.01 g
\item 収率 : 89.6 $\%$

\end{itemize}
\subsection*{考察}
\begin{itemize}
\item ろ紙についたベージュ色のペースト状の粉は反応式から考察するとNaClであると考えられる。
\item 結晶を濾取したあとはろ紙やデシケーターで十分乾燥する必要がある。乾燥が足りなかったために収率が高く、融点が低くなってしまった。

\end{itemize}
\subsection*{課題}

\subsubsection*{(1) 反応機構}
\pict{imgs4/hk.jpeg}{12}

\subsubsection*{(3) NaOHではなくEtONaを用いる理由}

$\rm OH^-$のような塩基をdiethyl 2-acetamidomalonate に作用させると色がついている3箇所で反応する可能性がある。このとき赤い色がついている炭素と反応すると$\rm OEt$が$\rm OH$と入れ替わってしまい目的の生成物ができくなってしまう。

\pict{imgs4/st1.jpeg}{10}

そこでNaOEtを用いることで赤や青の色がついた炭素で反応しても生成物が元と変わらないので、目的の生成物に必要な緑の炭素と反応させることができる。

\pict{imgs4/st2.jpeg}{10}

他に赤や青の炭素に作用させない手法としては、立体障害が大きい塩基を用いることで求核攻撃をさせない手法がある。


\section*{5日目}
\subsection*{結果}
\begin{itemize}
\item 加熱還流において溶液がはじめは白く濁っていたが115℃になったあたりで透明になった。
\item 撹拌後にpHが5 $\sim$ 6だったのでさらに塩酸を加えてpHを1にした。
\item 氷浴すると溶液が白く濁った。その後黄色いガム状の沈殿が得られた。
\item 内壁をスパーテルで擦るとガム状のものが白い粉に変わった。再結晶後に得られた結晶は透明度がある白い粉だった。

\end{itemize}
\subsubsection*{TLC}

\pict{imgs5/tlc.jpeg}{8}

\subsubsection*{再結晶}
\begin{itemize}
\item 重さ : 1.96 g
\item 収率 : 48.4$\%$
\item 融点 : 132℃, 134℃
\end{itemize}


\subsection*{考察}
\begin{itemize}
\item 酸性で再結晶した理由

酸性にすることで生成物のナトリウム塩を分離してNaClとして取り出すため。それにより生成物の水への溶解度が下がり、再結晶のときにより多くの結晶が得られる。
\end{itemize}


\subsection*{課題}
\subsubsection*{(1) 反応機構}
\pic{imgs5/hk.jpeg}





\section*{6日目}

\subsection*{結果}
\begin{itemize}
\item 油温が110℃を超えたあたりで白く濁っていた溶液が透明になった。
\item 反応溶液を減圧留去すると白色の粉が得られた。
\item 水を加えると黄色く濁った溶液となったが、白い粉が完全には溶けきらなかった。

\end{itemize}
\subsubsection*{TLC}
\pict{imgs6/tlc.jpeg}{8}

\subsubsection*{結晶(DL-phenylalanine)}
\begin{itemize}
\item 重さ : 0.18 g
\item 収率 : 22.4 $\%$
\item dec pt : 250℃, 251℃

\end{itemize}
\subsection*{考察}
\subsubsection*{エタノールを用いて再結晶した理由}
フェニルアラニンが双性イオンであるために水によく溶ける。そこで、水と混合可能でかつアミノ酸をとかしにくいエタノールで再結晶することで双性イオンのアミノ酸を得ることができる。

\subsubsection*{分解点(decomposition point)について}
今回用いたフェニルアラニンは双性イオンであるためにイオン結合が非常に強い。そのため融点が非常に高くなる。今回の融点測定管を用いた測定では250℃付近でフェニルアラニンが黄色の液状になったが、これはフェニルアラニンが分解し別の化合物になったからである。つまりフェニルアラニンは加熱するだけでは融点に達するより前に別の化合物になってしまうのである。分解点は融点と異なり化合物の構造が変化するため、冷ましても元の結晶に戻ることはない。

\subsection*{課題}

\subsubsection*{(1) 反応機構}
\pict{imgs6/hk.jpeg}{14}

\newpage

\subsubsection*{(4) phenylalanineのD体,L体を選択的に作る手法を提案する}

\small

The synthesis of tricyclic iminolactone (7) from (1R)-(+)-cam-phorquinone is outlined in Scheme beneath.

\pict{imgs-k/hk1.jpeg}{10}

\begin{itemize}
\mon{a} Selective acetalization of the less hindered carbonyl group was  accomplished  using ethylene glycol and a catalytic amount of p-toluenesulfonic acid in benzene with azeotropic removal of water to afford the desired monoacetal 2 in 78\% yield.

\mon{b} The remaining carbonyl group was then reduced with sodium borohydride at 0 ℃ followed by the removal of the acetal with aqueous sulfuric acid to give the 2-exo-hydroxyepicamphor (3) as the sole isomer in 94\% isolated yield.

\mon{c} Hydroxyketone 4 was then treated with N-Cbz-glycine, DCC, and DMAP in dry THF to furnish ester 5 in quantitative yield after column chromatography. 
mon{d} The Cbz group was removed by hydrogenolysis in absolute ethanol under a hydrogen atmosphere (1 atm) using 5\% palladium on carbon as catalyst for 14 hours. 

\mon{e} Concomitant cyclization to the imine occurred simultaneously during hydrogenation to give rise to the desired chiral template 7 in 76\% isolated yield over two steps.
\end{itemize}

\pict{imgs-k/hk2.jpeg}{10}

Alkylation of the tricyclic iminolactone 7 was carried out at -78 ℃ using various combinations of different bases, solvent systems, additives, and electrophiles. The results of the alkylation reactions of the enolate of 7 were strongly dependent on the reaction conditions employed. 

The following table summarizes the reaction conditions and results for introducing the $\rm PhCH_2$ group.

\begin{table}[H]
\begin{center}
\begin{tabular}{|c|c|c|c|c|c|c|}
\hline
entry & solvent  & base   & E+        & yield$(\%)^a$ & endo/exo$^b$          & \%de            \\ \hline
1     & THF/HMPA & KOBut  &  $\rm C_6H_5CH_2Br$  & $85^c$        & 1.2:1.3            &             \\ \hline
2     & THF      & KOBut  & $\rm C_6H_5CH_2Br$ & $58^d$        & 1:\textgreater{}99 & \textgreater{}98 \\ \hline
3     & THF/HMPA & n-BuLi & $\rm C_6H_5CH_2Br$ & 51 $(75)^e$   & \textgreater{}99:1 & \textgreater{}98 \\ \hline
4     & THF      & LDA    & $\rm C_6H_5CH_2Br$ & 59 $(81)^e$   & \textgreater{}99:1 & \textgreater{}98 \\ \hline
5     & THF/HMPA & LDA    & $\rm C_6H_5CH_2Br$ & 83 $(96)^e$   & \textgreater{}99:1 & \textgreater{}98 \\ \hline
\end{tabular}
\end{center}
\caption{(Table 1.)}
\end{table}

\footnotesize{
\begin{itemize}
\mon{a} The reported yields are isolated yields after column chromatography. 

\mon{b} The ratios were estimated by NMR integrations of the crude reaction mixtures on a Varian Mercury-400 NMR spectrometer.

\mon{c} The yield is composed of endo, exo, and dialkylated products in a ratio of 1.2:1.3:1. 

\mon{d} The yield is composed of exo and dialkylated products in a 1.4:1 ratio. 

\mon{e} The yields in parentheses are based on recovered starting material.
\end{itemize}}

\

Hydrolysis of the alkylated iminolactone for 2 hours 16 minutes in 8N hydrochloric acid solution at 87 ℃. gives the corresponding D-R-amino acids in good yield and enantiomeric excess.

\pict{imgs-k/hk3.jpeg}{12}

Therefore, considering Table 1, D-phenylalanine is obtained at the highest rate at entry 5, and L-phenylalanine is obtained at the highest rate at entry 2.


\subsubsection*{Referense thesis}
Chiral Tricyclic Iminolactone Derived from (1R)-(+)-Camphor as a Glycine Equivalent for the Asymmetric Synthesis of r-Amino Acids


\end{document}

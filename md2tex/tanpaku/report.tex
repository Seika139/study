\documentclass[a4paper,papersize,dvipdfmx]{jsarticle}
\usepackage{ascmac}
\usepackage{mathtools, amssymb,bm}
\usepackage{comment}
\usepackage[hiresbb]{graphicx}
\usepackage{tcolorbox,color}
\usepackage{here}
\usepackage{multirow} % 表でセルを結合しているときに必要
\usepackage{booktabs} % 表の形式をbooktabにすると必要
\usepackage{listings}
\tcbuselibrary{raster,skins,breakable}

\newcommand{\pic}[1]{\begin{center} \includegraphics[width=1.0\linewidth,clip]{#1} \end{center}}   %写真用
\newcommand{\pict}[2]{\begin{center} \includegraphics[width= {#2} cm]{#1} \end{center}}   %写真用

\newcommand{\piccap}[3]{\begin{center} \includegraphics[width= {#2} cm]{#1} \caption{#3} \label{fig {#1}} \end{center}} %キャプションつき画像
\newcommand{\redunderline}[1]{\textcolor{red}{\underline{¥textcolor{black}{#1}}}}   %赤いアンダーライン
\newcommand{\mon}[1]{\item[({#1})] \ }
\newcommand{\ctext}[1]{\raise0.2ex\hbox{\textcircled{\scriptsize{#1}}}}%文字を丸囲みする(2桁の数字までならいける)

%ここからソースコードの表示に関する設定
% 参考 https://qiita.com/ta_b0_/items/2619d5927492edbb5b03
\lstset{
  basicstyle={\ttfamily},
  identifierstyle={\small},
  commentstyle={\smallitshape},
  keywordstyle={\small\bfseries},
  ndkeywordstyle={\small},
  stringstyle={\small\ttfamily},
  frame={tb},
  breaklines=true,
  columns=[l]{fullflexible},
  numbers=left,
  xrightmargin=0zw,
  xleftmargin=3zw,
  numberstyle={\scriptsize},
  stepnumber=1,
  numbersep=1zw,
  lineskip=-0.5ex
}

% ソースコードを入れる方法
%\begin{lstlisting}[caption=hoge,label=fuga]
% #include<stdio.h>
% int main(){
%    printf("Hello world!");
% }
% \end{lstlisting}


% 複数図を横に並べるときのヒント http://wright.mydns.jp/?p=704
% \begin{figure}{H}
% \centering
% \begin{minipage}{0.22\hsize}
% \piccap{}{}{}
% \end{minipage}
% \begin{minipage}{0.06\hsize}
% \hspace{2mm}
% \end{minipage}
% \begin{minipage}{0.22\hsize}
% \piccap{}{}{}
% \end{minipage}
% \end{figure}

% 画像を貼る時はjpgかjpegで、pngはうまくいかない時もある

%\itemを四角で囲った数字にする場合は以下のコメントアウトを消す
%\renewcommand{\labelenumi}{\textbf{\framebox[1.5zw]{\theenumi}}}


%enumerateの2階層めのカウンタを1,2,3, にする時は以下のコメントアウトを消す
\renewcommand{\theenumii}{\arabic{enumii}}

%enumerateのカウンタについては以下を参照
% http://www3.otani.ac.jp/fkdsemi/pLaTeX_manual/kajyo.html


%enumerateの番号の出力形式を変更するには、カウンタの値を出力する命令を定義し直す。
%レベル	カウンタ	出力する命令	デフォルトの出力
%1	enumi	¥theenumi	アラビア数字(1,2,3,・・・)
%2	enumii	¥theenumii	小文字のアルファベット(a,b,c,・・・)
%3	enumiii	¥theenumiii	小文字のローマ数字(小文字のローマ数字(ⅰ,ⅱ,ⅲ,・・・)
%4	enumiv	¥theenumiv	大文字のアルファベット(A,B,C,・・・)
%例:¥enumiカウンタを大文字のローマ数字で出力する設定
% ¥renewcommand{¥theenumi}{¥Roman{enumi}}

% 番号の出力形式
%命令	出力形式
%¥arabic	アラビア数字(1、2、3、・・・)
%¥roman	ローマ数字(ⅰ、ⅱ、ⅲ、・・・)
%¥Roman	ローマ数字(Ⅰ、Ⅱ、Ⅲ、・・・)
%¥alph	アルファベット(a、b、c、・・・)
%¥Alph	アルファベット(A、B、C、・・・)

% ページ番号を消す場合は以下のコメントアウトを消す
%\pagestyle{empty}

\begin{document}

\title{蛋白質代謝学教室}
\author{10191043 鈴木健一 10班}
\date{}
\maketitle

%以下の3つからフォントサイズを選択するとよい
%\footnotesize
%\small
%\normalsize


\part*{遺伝子破壊実験}

\section*{目的}
CDC26遺伝子破壊株を作成し、その表現系を観察する。

\section*{方法}
実習書に則って行った。

\section*{結果}

\piccap{img1/edo.jpg}{8}{\\電気泳動の結果}

\subsection*{顕微鏡観察}

28度で一晩培養したものは以下のようになった。

\begin{figure}[H]
\begin{center}
\begin{tabular}{c}

\begin{minipage}{0.22\hsize}
\piccap{img1/28-42.jpg}{4}{サンプル42}
\end{minipage}

\begin{minipage}{0.06\hsize}
\hspace{2mm}
\end{minipage}

\begin{minipage}{0.22\hsize}
\piccap{img1/28-43.jpg}{4}{サンプル43}
\end{minipage}

\begin{minipage}{0.06\hsize}
\hspace{2mm}
\end{minipage}

\begin{minipage}{0.22\hsize}
\piccap{img1/28-44.jpg}{4}{サンプル44}
\end{minipage}

\\

\begin{minipage}{0.06\hsize}
\vspace{10mm}
\end{minipage}

\\

\begin{minipage}{0.22\hsize}
\piccap{img1/28-cdc11.jpg}{4}{cdc11}
\end{minipage}

\begin{minipage}{0.06\hsize}
\hspace{2mm}
\end{minipage}

\begin{minipage}{0.22\hsize}
\piccap{img1/28-cdc15-2.jpg}{4}{cdc15-2}
\end{minipage}

\begin{minipage}{0.06\hsize}
\hspace{2mm}
\end{minipage}

\begin{minipage}{0.22\hsize}
\piccap{img1/28-cdc20.jpg}{4}{cdc20}
\end{minipage}

\\

\begin{minipage}{0.06\hsize}
\vspace{10mm}
\end{minipage}

\\

\begin{minipage}{0.22\hsize}
\piccap{img1/28-cdc28.jpg}{4}{cdc28}
\end{minipage}

\begin{minipage}{0.06\hsize}
\hspace{2mm}
\end{minipage}

\begin{minipage}{0.22\hsize}
\piccap{img1/28-cdc34.jpg}{4}{cdc34}
\end{minipage}

\begin{minipage}{0.06\hsize}
\hspace{2mm}
\end{minipage}

\begin{minipage}{0.22\hsize}
\piccap{img1/28-wt.jpg}{4}{WT}
\end{minipage}

\end{tabular}
\end{center}
\end{figure}

37度で一晩培養したものは以下のようになった。

\begin{figure}[H]
\begin{center}
\begin{tabular}{c}

\begin{minipage}{0.22\hsize}
\piccap{img1/37-42.jpg}{4}{サンプル42}
\end{minipage}

\begin{minipage}{0.06\hsize}
\hspace{2mm}
\end{minipage}

\begin{minipage}{0.22\hsize}
\piccap{img1/37-43.jpg}{4}{サンプル43}
\end{minipage}

\begin{minipage}{0.06\hsize}
\hspace{2mm}
\end{minipage}

\begin{minipage}{0.22\hsize}
\piccap{img1/37-44.jpg}{4}{サンプル44}
\end{minipage}

\\

\begin{minipage}{0.06\hsize}
\vspace{10mm}
\end{minipage}

\\

\begin{minipage}{0.22\hsize}
\piccap{img1/37-cdc11.jpg}{4}{cdc11}
\end{minipage}

\begin{minipage}{0.06\hsize}
\hspace{2mm}
\end{minipage}

\begin{minipage}{0.22\hsize}
\piccap{img1/37-cdc15-2.jpg}{4}{cdc15-2}
\end{minipage}

\begin{minipage}{0.06\hsize}
\hspace{2mm}
\end{minipage}

\begin{minipage}{0.22\hsize}
\piccap{img1/37-cdc20.jpg}{4}{cdc20}
\end{minipage}

\\

\begin{minipage}{0.06\hsize}
\vspace{10mm}
\end{minipage}

\\

\begin{minipage}{0.22\hsize}
\piccap{img1/37-cdc28.jpg}{4}{cdc28}
\end{minipage}

\begin{minipage}{0.06\hsize}
\hspace{2mm}
\end{minipage}

\begin{minipage}{0.22\hsize}
\piccap{img1/37-cdc34.jpg}{4}{cdc34}
\end{minipage}

\begin{minipage}{0.06\hsize}
\hspace{2mm}
\end{minipage}

\begin{minipage}{0.22\hsize}
\piccap{img1/37-wt.jpg}{4}{WT}
\end{minipage}

\end{tabular}
\end{center}
\end{figure}

\section*{考察・課題}

\begin{enumerate}
\mon{1} CDC26は細胞周期後期促進因子で、細胞周期によって制御されるユビキチンタンパク質リガーゼとして機能する細胞周期後期促進複合体(APC)の構成要素である Saccharomyces cerevisiae Cdc26と非常によく似ていおり、タンパク質分解を担う。

\mon{2}
導入遺伝子の末端から約20bpを認識するプライマーを用いることで、遺伝子が導入できたもののみにバンド生じさせることができる。
導入は成功しているが、酵素が不活性したためにバンドが生じなかったという現象を考慮してpositive controlを用意する必要がある。
\end{enumerate}

\part*{EMSによる突然変異導入実験}

\section*{目的}
野生型株を変異原処理してよりカナバニン耐性となった変異株を取得する。また、変異原処理の効率を大まかに検定する。

\section*{方法}
実習書と配布プリントに従って行った。偶数班のため、用いた株はBY4743株(二倍体)である。

\section*{結果}

YDP培地に1/40kで希釈した懸濁液をインキュベートしたものの結果は以下のようになった。

\begin{table}[H]
  \centering
\begin{tabular}{@{}lllll@{}}
\toprule
& 9班(一倍体) & 10班(二倍体) & 奇数班平均 & 偶数班平均         \\ \midrule
EMS-    & 437      & 443   & 371.6 & 334.625 \\
EMS+    & 254      & 253   & 294.4 & 273.5   \\ \bottomrule
\end{tabular}
\end{table}

1/200希釈、カナバニン培地で培養後のコロニー数(EMS処理直後)の結果は以下のようになった。

\begin{table}[H]
  \centering
\begin{tabular}{@{}lllll@{}}
\toprule
& 9班(一倍体) & 10班(二倍体) & 奇数班平均 & 偶数班平均    \\ \midrule
EMS-    & 0        & 0     & 0.111 & 0.5 \\
EMS+    & 0        & 0     & 0     & 0   \\ \bottomrule
\end{tabular}
\end{table}

1/200希釈、カナバニン培地で培養後のコロニー数(EMS処理1日後)の結果は以下のようになった。
10班はカナバニンYPDの方は1日培養し忘れたため、増殖が十分ではないと思われる。

\begin{table}[H]
  \centering
\begin{tabular}{@{}lllll@{}}
\toprule
& 9班(一倍体) & 10班(二倍体) & 奇数班平均 & 偶数班平均  \\ \midrule
EMS-    & 0        & 0     & 8.2   & 0.1 \\
EMS+    & 54       & 0     & 164   & 9   \\ \bottomrule
\end{tabular}
\end{table}


\section*{考察・課題}
\begin{enumerate}
\mon{1} 一倍体の平均での致死率は$20.8\%$、二倍体の平均での致死率は$18.3\%$となった。
\mon{2} カナバニンを取りこむトランスポーターやカナバニンをリボソームに運ぶtRNAの遺伝子に変異が入るとカナバニン耐性を獲得すると考えられる。
\mon{3} 一倍体の平均はEMS処理後一日置いた状態でEMS-がコロニー数8.2、EMS+が164個のコロニーを形成した。一方、二倍体はいずれもコロニー数はそれぞれ0.1と9であった。一倍体では変異が表現系に現れやすいのに対し、二倍体は2つの遺伝子に変異がないと表現系に現れないため、耐性株が出現しにくい。
\mon{4} EMS処理後すぐに培地にまいたカナバニン耐性株はコロニーが0個であったが、一晩培養すると164個のコロニーを形成した。このことから、EMS処理によるカナバニン耐性能獲得には時間がかかることが分かる。
\end{enumerate}

\part*{Two-hybrid システムを用いた実験}

\section*{目的}
Two-hybrid システムの原理を理解する。プロテアソームサブユニット間の結合を Two-hybrid システムで検出する。

\section*{方法}
実習書に則って行った。

\section*{結果}

\begin{table}[H]
  \centering
\begin{tabular}{|l|l|l|l|l|l|l|l|l|l|}
\hline
GAD\textbackslash{}GBD & Rpt1 & Rpt2 & Rpt3 & Rpt4 & Rpt5 & Rpt6 & Rpn1 & Rpn2 & Rpn13 \\ \hline
Rpt1                   & +++  & +++  & +   & ++   & ++   & ++   & ++   & ++   & +++   \\ \hline
Rpt2                   & +++  & +++  & +    & +    & -    & ++   & ++   & ++   & +++   \\ \hline
Rpt3                   & +++  & +++  & +   & ++   & ++  & ++   & -    & ++   & +++   \\ \hline
Rpt4                   & +++  & +++  & ++   & ++   & ++   & ++   & ++   & ++   & +++   \\ \hline
Rpt5                   & +++  & +++  & ++   & ++   & ++   & ++   & ++   & ++   & ++   \\ \hline
Rpt6                   & +++  & +++  & +++  & ++   & ++   & ++   & ++  & ++   & +++   \\ \hline
Rpn1                   & +++  & +++  & +    & +   & ++  & ++  & +    & ++   & +++   \\ \hline
Rpn2                   & +++  & +++  & ++   & ++   & ++   & ++   & +    & ++   & +++   \\ \hline
Rpn13                  & +++  & +++  & ++   & ++   & ++   & ++   & ++   & +++  & +++   \\ \hline
\end{tabular}
\end{table}

表の記号はそれぞれ以下のように対応している。

\begin{table}[H]
  \centering
\begin{tabular}{|l|l|}
\hline
-   & Hisでも全く生えない            \\ \hline
+   & Hisでも生えているのか?          \\ \hline
++  & しっかりHis+, Adeは生えているのか? \\ \hline
+++ & しっかりHis+とAde+と両方生える    \\ \hline
\end{tabular}
\end{table}

\section*{考察・課題}

\begin{enumerate}
\mon{1} 実際のプロテアソームは以下のサイトにある画像を参考にしている。

(https://www.jst.go.jp/pr/info/info219/zu3.html)

特に陰性であったRpt2-5と3-1は実際に離れた位置にあることがわかる。
\pict{img3/real.png}{7}

\mon{2} Two-hybridシステムは感度が高いことをはじめとした多くの利点を持っているが、欠点も存在する。例えば、タンパク自身が転写活性を持つ場合や転写活性化ドメインを付けたタンパク質がDNA結合能には正確な評価ができない。また、哺乳類の蛋白質を酵母内でアッセイすると、正確にタンパクの折り畳みができない、他の会合に必要な因子が存在しないなどの理由で、実際には相互作用する場合でも、酵母内では相互作用しない場合がある。
\end{enumerate}

\part*{変異体ライブラリを用いた網羅的スクリーニング}

\section*{目的}
ラパマイシンとはどんな薬剤で、どのような遺伝子産物の変異がラパマイシン感受性となるか調べる。

\section*{方法}
実習書に則って行った。

\section*{結果}
\begin{table}[H]
  \centering
\begin{tabular}{|c|c|c|c|c|c|c|cccc}
\cline{1-3} \cline{5-7} \cline{9-11}
ORF     & GENE    & 感受性 &  & ORF     & GENE  & 感受性 & \multicolumn{1}{c|}{} & \multicolumn{1}{c|}{ORF}     & \multicolumn{1}{c|}{GENE}    & \multicolumn{1}{c|}{感受性} \\ \cline{1-3} \cline{5-7} \cline{9-11}
YGL169W & SUA5    & +   &  & YGR074W & SMD1  & +   & \multicolumn{1}{c|}{} & \multicolumn{1}{c|}{YGR190C} & \multicolumn{1}{c|}{}        & \multicolumn{1}{c|}{+}   \\ \cline{1-3} \cline{5-7} \cline{9-11}
YGL171W & ROK1    & ±   &  & YGR083C & GCD2  & ー   & \multicolumn{1}{c|}{} & \multicolumn{1}{c|}{YGR191W} & \multicolumn{1}{c|}{HIP1}    & \multicolumn{1}{c|}{+}   \\ \cline{1-3} \cline{5-7} \cline{9-11}
YGL201C & MCM6    & ±   &  & YGR090W & UTP22 & +   & \multicolumn{1}{c|}{} & \multicolumn{1}{c|}{YGR198W} & \multicolumn{1}{c|}{YGR198W} & \multicolumn{1}{c|}{+}   \\ \cline{1-3} \cline{5-7} \cline{9-11}
YGL225W & VRG4    & +   &  & YGR091W & PRP31 & +   & \multicolumn{1}{c|}{} & \multicolumn{1}{c|}{YGR211W} & \multicolumn{1}{c|}{ZPR1}    & \multicolumn{1}{c|}{±}   \\ \cline{1-3} \cline{5-7} \cline{9-11}
YGL233W & SEC15   & +   &  & YGR094W & VAS1  & +   & \multicolumn{1}{c|}{} & \multicolumn{1}{c|}{YGR251W} & \multicolumn{1}{c|}{YGR251W} & \multicolumn{1}{c|}{+}   \\ \cline{1-3} \cline{5-7} \cline{9-11}
YGL238W & CSE1    & +   &  & YGR095C & RRP46 & +   & \multicolumn{1}{c|}{} & \multicolumn{1}{c|}{YGR264C} & \multicolumn{1}{c|}{MES1}    & \multicolumn{1}{c|}{+}   \\ \cline{1-3} \cline{5-7} \cline{9-11}
YGL239C &         & +   &  & YGR099W & TEL2  & ー   & \multicolumn{1}{c|}{} & \multicolumn{1}{c|}{YGR265W} & \multicolumn{1}{c|}{}        & \multicolumn{1}{c|}{+}   \\ \cline{1-3} \cline{5-7} \cline{9-11}
YGL245W & GUS1    & +   &  & YGR113W & DAM1  & +   & \multicolumn{1}{c|}{} & \multicolumn{1}{c|}{YGR267C} & \multicolumn{1}{c|}{FOL2}    & \multicolumn{1}{c|}{+}   \\ \cline{1-3} \cline{5-7} \cline{9-11}
YGR013W & SNU71   & ー   &  & YGR114C &       & +   & \multicolumn{1}{c|}{} & \multicolumn{1}{c|}{YGR278W} & \multicolumn{1}{c|}{CWC22}   & \multicolumn{1}{c|}{+}   \\ \cline{1-3} \cline{5-7} \cline{9-11}
YGR030C & POP6    & ー   &  & YGR120C & COG2  & +   & \multicolumn{1}{c|}{} & \multicolumn{1}{c|}{YHL015W} & \multicolumn{1}{c|}{RPS20}   & \multicolumn{1}{c|}{+}   \\ \cline{1-3} \cline{5-7} \cline{9-11}
YGR046W & YGR046W & +   &  & YGR140W & CBF2  & +   & \multicolumn{1}{c|}{} & \multicolumn{1}{c|}{YHR005C} & \multicolumn{1}{c|}{GPA1}    & \multicolumn{1}{c|}{+}   \\ \cline{1-3} \cline{5-7} \cline{9-11}
YGR048W & UFD1    & +   &  & YGR145W & ENP2  & ±   & \multicolumn{1}{c|}{} & \multicolumn{1}{c|}{YHR007C} & \multicolumn{1}{c|}{ERG11}   & \multicolumn{1}{c|}{+}   \\ \cline{1-3} \cline{5-7} \cline{9-11}
YGR060W & ERG25   & +   &  & YGR175C & ERG1  & +   & \multicolumn{1}{c|}{} & \multicolumn{1}{c|}{YHR036W} & \multicolumn{1}{c|}{YHR036W} & \multicolumn{1}{c|}{ー}   \\ \cline{1-3} \cline{5-7} \cline{9-11}
YGR065C & VHT1    & +   &  & YGR185C & TYS1  & +   & \multicolumn{1}{c|}{} & \multicolumn{1}{c|}{YHR040W} & \multicolumn{1}{c|}{BCD1}    & \multicolumn{1}{c|}{+}   \\ \cline{1-3} \cline{5-7} \cline{9-11}
YGR073C &         & +   &  & YGR186W & TFG1  & ±   &                       &                              &                              &                          \\ \cline{1-3} \cline{5-7}
\end{tabular}
\end{table}


\section*{考察・課題}
\begin{enumerate}
\mon{1} 感受性を強く示した株の遺伝子を調べた。
\begin{itemize}
\item YGR013W : スプライソソームを介したmRNAスプライシングに必要なU1 snRNPの成分。
\item YGR030C : RNase MRP、核RNase Pおよびテロメラーゼのサブユニット。
\item YGR083C : 翻訳開始因子eIF2Bのデルタサブユニット。
\item YGR099W : クロマチンリモデリングに関与するASTRA複合体のサブユニット。
\item YHR036W : 脂質恒常性におけるApq12pおよびBrr6pと相互作用する。
\end{itemize}

ラパマイシン(別名シロリムス)は下記のような構造からなる薬剤でリンパ脈管筋腫症や結節性硬化症に伴う皮膚病変に対して効果がある。
\pict{img4/rpm.png}{5}
いくつかの遺伝子について調査したが、ラパマイシンはAktとeIF両方のリン酸化も増加させることから、
YGR083Cの発現減弱がラパマイシン感受性と関連があることが想定される。

\mon{2} タンパク質の分解において、プロテアソームがユビキチン化した標的蛋白質を選択的に分解するのに対し、
オートファジーでは、必要なときにだけあらわれるオートファゴソームという一過性のオルガネラ空間の中がリソソーム酵素によって分解する。
オートファジーはアポトーシスが不完全な時にラパマイシンによって誘導される。

\end{enumerate}

\subsection*{参考資料}
\begin{itemize}
\item https://www.dojindo.co.jp/letterj/104/reviews02.html

\item https://cancerres.aacrjournals.org/content/65/16/7052.full

\item https://www.natureasia.com/ja-jp/jobs/tokushu/detail/287

\item https://www.yeastgenome.org/locus/S000001078
\end{itemize}

\end{document}

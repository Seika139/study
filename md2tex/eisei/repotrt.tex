\documentclass[a4paper,papersize,dvipdfmx]{jsarticle}
\usepackage{ascmac}
\usepackage{mathtools, amssymb,bm}
\usepackage{comment}
\usepackage[hiresbb]{graphicx}
\usepackage{tcolorbox,color}
\usepackage{here}
\tcbuselibrary{raster,skins,breakable}

\newcommand{\pic}[1]{\begin{center} \includegraphics[width=1.0\linewidth,clip]{#1} \end{center}}   %写真用
\newcommand{\pict}[2]{\begin{center} \includegraphics[width= {#2} cm]{#1} \end{center}}   %写真用
\newcommand{\piccap}[3]{\begin{figure}[H] \centering \includegraphics[width= {#2} cm]{#1} \caption{#3} \label{fig {#1}} \end{figure}} %キャプションつき画像
\newcommand{\redunderline}[1]{\textcolor{red}{\underline{¥textcolor{black}{#1}}}}   %赤いアンダーライン
\newcommand{\mon}[1]{\item[({#1})] \ }
\newcommand{\ctext}[1]{\raise0.2ex\hbox{\textcircled{\scriptsize{#1}}}}%文字を丸囲みする(2桁の数字までならいける)

% 画像を貼る時はjpgかjpegで、pngはうまくいかない時もある

%\itemを四角で囲った数字にする場合は以下のコメントアウトを消す
%\renewcommand{\labelenumi}{\textbf{\framebox[1.5zw]{\theenumi}}}


%enumerateの2階層めのカウンタを1,2,3, にする時は以下のコメントアウトを消す
\renewcommand{\theenumii}{\arabic{enumii}}

%enumerateのカウンタについては以下を参照
% http://www3.otani.ac.jp/fkdsemi/pLaTeX_manual/kajyo.html


%enumerateの番号の出力形式を変更するには、カウンタの値を出力する命令を定義し直す。
%レベル	カウンタ	出力する命令	デフォルトの出力
%1	enumi	¥theenumi	アラビア数字(1,2,3,・・・)
%2	enumii	¥theenumii	小文字のアルファベット(a,b,c,・・・)
%3	enumiii	¥theenumiii	小文字のローマ数字(小文字のローマ数字(ⅰ,ⅱ,ⅲ,・・・)
%4	enumiv	¥theenumiv	大文字のアルファベット(A,B,C,・・・)
%例:¥enumiカウンタを大文字のローマ数字で出力する設定
% ¥renewcommand{¥theenumi}{¥Roman{enumi}}

% 番号の出力形式
%命令	出力形式
%¥arabic	アラビア数字(1、2、3、・・・)
%¥roman	ローマ数字(ⅰ、ⅱ、ⅲ、・・・)
%¥Roman	ローマ数字(Ⅰ、Ⅱ、Ⅲ、・・・)
%¥alph	アルファベット(a、b、c、・・・)
%¥Alph	アルファベット(A、B、C、・・・)




\begin{document}

\title{薬学実習4 衛生化学教室}
\author{10191043 鈴木健一}
%作成日を入れる場合は消す
\date{}
\maketitle

%以下の3つからフォントサイズを選択するとよい
%\footnotesize
%\small
%\normalsize


\part*{実習1 ヒツジ赤血球膜からの膜脂質の単離}

\section*{目的}
ヒツジ赤血球膜から全脂質を抽出、同定し特定のリン脂質(ホスファチジルセリン、ホスファチジルエタノールアミン)を単離する。

\section*{方法}

\subsection*{ヒツジ赤血球膜より全脂質のBligh-Dyer法による抽出}

ヒツジ赤血球膜分画8mlを100mlの三角フラスコに移す。ついでメタノール20ml、クロロホルム10mlを、この順に加える。
クロロホルム/メタノール/水の比は1:2:0.8(v/v)であり均一になっている。
30分室温にてマグネチックスターラーで攪拌し抽出を行う。
次に10mlのクロロホルム、10mlのイオン交換水を加え5分間激しく撹拌する。この時点で、クロロホルム/メタノール/水の比は1:1:0.9になっており、2層に分離する。
これを遠沈管2本に移し、2,500rpmで5分間遠心する。
遠心終了後、下層のみをパスツールピペットと安全ピペッターを用いて100mlの三角フラスコに移す。
このとき中間層を取らないように注意する。
下層をひだ折り濾紙を用いて濾過し、エバポレーター用50ml丸底フラスコに取る。
エバポレーターにて溶媒を除去する。
このとき水浴の温度が35℃以上にならないように注意する。
ほとんど溶媒がなくなったら、0.1ml未満べンゼンを加え(1滴で十分)、水を共沸させて除く。
次に5mlのクロロホルムにとかし、先細スクリリューキャップチューブに保存する
(過酸化を防止するためにまわりをアルミホイルで包む)。

\subsection*{薄層クロマトグラフィー(Thin LaYer chromatography, TLC)}
展開溶媒(クロロホルム:メタノール:酢酸:水=60:30:10:5)を50ml程度作り、展開槽に充填しておく。
三角フラスコにクロロホルム$\rightarrow$メタノール$\rightarrow$酢酸$\rightarrow$水の順番に加え、
それぞれを加えた後によく撹拌して混ぜる。展開槽のなかに濾紙をたてて、槽中に溶媒の蒸気が飽和するようにする。
なるべく早めにつくり、溶媒の蒸気を飽和させておく。
展開中はできるだけふたを開けないようにする。

ヒツジ赤血球膜脂質0.1m1およびホスフアチジルエタノールアミン(PE)、ホスフアチジルセリン(PS)、
スフィンゴミエリン(SM)の各標品(5mM)をシリカゲルプレート(10$\times$5cm)2枚にスポットする。
ヒツジ赤血球膜脂質は0.1mlを小試に計り取り、全量をスポットする。
スポットがあまり大きく広がらないように少しずつ乾かしながら行う。
標品は2-3回のスポットでよい。クロロホルム/メタノール/酢酸/水(60:30:10:5)の溶媒系にて展開する。
展開後、風乾し、1枚はヨード、もう1枚はニンヒドリン発色を行い、各スポットの同定を行う。

\subsection*{薄層クロマトグラフィー}

展開溶媒(クロロホルム/メタノール/酢酸/水=60:30:10:5)を50ml程度作り、展開槽に充填する。
展開槽のなかには濾紙をたてる。なるべく早めにつくり、30分以上溶媒の蒸気を飽和させる。
昨日得たヒツジ赤血球膜全脂質分画が入った先細スクリューキャップチューブをエバポレーターにかけ、
溶媒を留去する(接続には白いコネクターを用いる)。
クロロホルムを0.5ml加え良くボルテックスし、全量をシリカゲルプレート(10$\times$10cm)に5cm幅でスポットする。
スポットがあまり大きく広がらないように少しずつ乾かしながら行う。
両端にPEとPSの標品を2ー3回スポットし、展開する。
展開後、風乾し、ヨード発色によりみられたPEとPSのスポットを鉛筆でマークする。


\subsection*{PEとPSのスポットのかき取り}

アルミホイルを敷き、PEのスポット部分のシリカゲルをクロロホルムで良く洗浄したスパーテルを用いてかき取る。
かき取ったシリカゲルは薬包紙に集め、5mlのメタノールが入った遠沈管(50ml)に入れる。
PEと同様にPSもかき取り、PEとは別の遠沈管に回収する。

\subsection*{シリカゲルからのPE、PSのBligh-Dyer法による抽出}

5mlのメタノールとシリカゲルが入った遠沈管にクロロホルム2.5mlとイオン交換水2mlを加え5分間ボルテックスする。
次にクロロホルム2.5mlとイオン交換水2.5mlを追加し、5分間ボルテックスする。
2,500rpmで5分間遠心後、下層のみをパスツールピペットと安全ピペッターを用いて50mlの三角フラスコに移す。
このとき中間層を取らないように注意する。下層をひだ折り濾紙を用いて濾過し、
エバポレーター用50ml丸底フラスコに取る(ひだ折り濾紙の上からクロロホルム10mlを加え、洗い
こむ)。エバポレーターにて溶媒を除去し、ほとんど溶媒がなくなったら、べンゼンを加え(1滴で十
分)、水を共沸させて除く。5mlのクロロホルムにとかした後、
小試に移し、サランラップをまいたゴム栓をはめて保存する(過酸化を防止するために小試のまわりをアルミホイルで包む)。

\subsection*{PS、PE画分のリン脂質量の定量}

PS,PE画分とPSおよびPEの標準リン脂質溶液(0.1mM)を下表のようにクロロホルムで希釈した溶液を作り、
塩化鉄ーチオシアン酸アンモニウム溶液(鉄チオシアン酸試薬)を0.5mlずっ加える。1分間よく撹拌(ボルテックス)したあと、
遠心機でスピンダウンする。クロロホルム層(下層)をパスツールピペットで抽出し、488nmの吸光度を測定する。
標準リン脂質溶液から検量線を作製し、PS、PE画分のリン脂質量を定量する。




\section*{結果}

\subsection*{TLC}
ニンヒドリン発色ではPSとPEのスポットが現れた。
\piccap{images/nin.jpg}{5}{ニンヒドリン発色}
ヨード発色ではそれに加えてSMのスポットも見られた。
\piccap{images/yodo.jpg}{5}{ヨード発色}


翌日は赤血球膜全脂質分画についてTLCを行PSとPEの画分をそれぞれ取り出した。
その後標準液の濃度をもとに検量線を求め、それぞれの分画に含まれるPE,PSの濃度と物質量を求めた。
\pict{images/graph.png}{10}
検量線よりPSの濃度は0.0483mM、PEは0.1015mMと求められ、これをもとにTube Bに0.23ml、Tube Dに0.47mlのPEを分注した。


\section*{考察}
我々のTLCでは見られなかったが、ヨード発色ではPEのスポットよりもさらに上にコレステロールのスポットが見られるらしい。
また、検量線を作る際に用いたPE標準液のTube 8が操作中に一部失われてしまったため、PEの検量線が必ずしも正確であるとは言い切れない。

\section*{課題}

\subsection*{ヨード発色とニンヒドリン発色の原理}

\begin{itemize}
\item ニンヒドリン発色

ニンヒドリンがアミノ酸やタンパク質が持つアミノ基と以下のように反応することで青紫~赤紫色に呈色する。
\pict{images/hanou.png}{6}

\item ヨード発色
ヨウ素がヨウ素昇華することで有機化合物に物理的に吸着することで呈色される。
特に不飽和結合と相互作用しやすく有機化合物全般に対応している。
呈色後のTLCを放置するとヨウ素が抜けていくので、素早く印をつけておく必要がある。


\end{itemize}
\part*{実習2 LysoPSによるマスト細胞の活性化}

\section*{目的}
マウスからマスト細胞を抽出し立つ顆粒反応によってヒスタミンの定量を行う。

\section*{方法}
\subsection*{PLA2反応}

前日に調製したA,Bのスクリューキャップチューブをエバポレーターにかけ、クロロホルムを完全に留去する(接続には白いネクターを用いる)。
A,Bのチューブに750$\mu$l反応バッファー(0.1M Tris-HCl(pH8.0),0.1M CaCl2)を入れ、蓋をしたあと、
ソニケーターで超音波処理する(1分前後、超音波処理の掛かり具合に注意する)。
ピペットマンでA,Bのチ=ープにPLA2溶液50unit(約10$\mu$l)を加える(スタッフが行なう)。
さらにミリQ240ⅵを加え、蓋をして軽くボルテックスし、37℃で振りながら1時間インキュべートする。
待ち時間に展開溶媒(クロロホルム/メタノール/酢酸/水=50:30:8:4)50mlを作り、展開槽に充填しておく。
展開槽の内側にはろ紙を入れる。

\subsection*{反応液からの脂質抽出}

インキュべーション後、メスピペットでA,Bのチュープにメタノール2ml,クロロホルム1mlを入れてボルテックスする。
さらに、3N塩酸900$\mu$l,クロロホルム1mlを加えて5分間ボルテックスする。蓋を外し2500rpm室温で5分間遠心した後、
パスツールピペットを用いて下層を丁寧にとり(安全ピペッターを用い、界面を乱さないように注思する)、
新しいスクリューキャップチューブに回収する。残った上層にクロロホルム1mlを入れて、5分間ボルテックスし、
蓋を外し2500rpm室温で5分間遠心する。下層を抽出し、1回目の下層を回収したチューブに回収する。
下層を集めたチューブをエバポレーターにかけ、溶媒を留去したあと、パスツールピペット、べンゼンを1滴入れて、再びエバポレーターにかける。
溶媒を再び留去した後、メスピペットでメタノールとクロロホルムを100$\mu$lずつ入れ、30秒ボルテックスする。

\subsection*{TLCによる反応の確認}

C,Dのチューブの溶媒を除去し、メスピペットでメタノールとクロロホルムを100$\mu$lずつ入れ、30秒ボルテックスする。
A,Bのチチューブから40mlをキャビラリーでとり、TLCプレート(5cmx10cm)に実習書にあるようにスポットする(キャビラリーで20$\mu$lのラインまでとる。それを2回繰返す)。同様にC,Dのチューブから、80$\mu$lをキャピラリーでとり、TLCプレートにスポットする(量が多いので、ドライヤー(冷風)を当てながらスポットする)。
リゾPS,リゾPEの標品もスポットし(10$\mu$l)、乾燥させた後、展開槽につけて展開し、ニンヒドリン発色する。
リゾPS,リゾPEの生成を確認したら、4本すべてのスクリューキャップチューブの溶媒を除去し、班名、内容物を明記し提出する。

\subsection*{ラット腹腔マスト細胞の調製}
Wistar系雄ラットをイソフルラン麻酔下で頸動脈切断により脱血死させる(大学院生が行う)。
腹部の毛を濡らした後、皮膚を切開し、腹筋を露出させる(デモを見て下さい)。
約10mlの氷冷バッファー(0.01$\%$BSA-HBT)を腹腔に注射器で注入する(このとき、胃や腸または腹筋内に打ち込まないよう
に注意する。デモを見てください)。
腹部を約2分間ツサージする。マッサージは、ラットを仰向けに寝かせ、両脇腹に指を当てて軽く揉むように行い、
腹腔内のバッファーがよく撹拌されるようにする。
あまり強いと肝臓が破れて出血するので注意する。
腹筋表面についた毛を除き、腹筋中央部を縦に切開し(この時バッファーが外に漏れないように注意)、
ピペットで腹腔内のバッファーを氷上の15ml tubeに回収する。15ml tubeを、1,00rpmで5分間遠心し、上清をデカントで捨てる。
残った細胞に、新たに氷冷バッファー2mlを加えてピペットマンでサスペンドし、
均一になるまでほぐす(あまり激しいと細胞が痛む。また、泡を立てないように注意)


ピペットマンで50$\mu$1はかり取り、1.5ml tubeに入れる。1.5ml tubeにとった細胞懸濁液50$\mu$lにトルイジンカレー染色液50を加えてサスペンドし、
5分放置する。血球計算盤を用いて顕微鏡下で観察し、マスト細胞の個数を数える(マスト細胞は数分で紫色に染まる)。
15ml tubeに残っている細胞懸濁液に、8m1のバッファーを加え、再び1,000rPm5分間遠心する(遠心機待ちをする場合、氷上においておくこと)。
上清をデカンテーションで捨て、1ml中にマスト細胞が$2\times10^5$個存在するよう氷冷バッファーを加えてピペットマンでサスペンドする。
氷上においておく。


\subsection*{脂質調製}
スクリューキャップチューブ中で乾固したA:PS/PLA2反応後、B:PE/PLA2,反応後に600$\mu$l、C:PS、D:PE
に200$\mu$l,バッファー(0.01%BsA_HBT)を加えボルテックスする。
蓋をして、ソニケーターで約1分間超音波処理し、ボルテックスして脂質懸濁液を調製する。
1.5mlエッペンドルフチューブに脂質懸濁液の1/10希釈液を1ml作る(脂質懸濁液100$\mu$l、バッファー900$\mu$l。できた1/10希釈液をA1、B1、C1、
D1、とする。新しい1.5ml エッペンドルチューブを用意しA2、B2の1/10希釈液を1ml作る(1/50希釈液100$\mu$l+バッファー900$\mu$l。
できた、1/500希釈液をA3、B3とする。

以上の操作でA1:PS/PLA2反応後1/10希釈液、A2:PS/PLA2反応1/50希釈液、A3:PS/PLA2反応後1/500希釈液、B1:PE/PLA2反応後1/10希釈液、
B2:PE/PLA2反応後1/50希釈液、B3:PE/PLA2反応後1/500希釈液、C1:PS 1/10 希釈液、D1:PE1/10希釈液がエッペンドルチューブにできる。
また、4$\mu$MリゾPS溶液を1/10希釈する。これらを次の実験に用いる。


\subsection*{脱顆粒反応}

2mlチューブ(底が丸い)に、氷上で表5の通り試薬を入れる。水浴に入れる直前にNo.1~13に
マスト細胞懸濁液を50$\mu$lずつ加える。(マスト細胞は加える直前に転倒混和して均一にすること)
チューブホルダーにNo.1~14をセットし、37℃の水浴に入れ、15分間インキュべーション後、ホルダーごと氷上に回収し、
素早くそれぞれのチューブに氷冷したバッファー(0.01% BSA-HBT) 100$\mu$lを加える(素早く温度を下げ、できるだけ同時に脱顆粒反応を止める)。
遠心機まで氷上で運び、No.1~12を冷却遠心機にセットし3000rpm、4℃で5分間遠心する。(遠心機待ちの場合は氷上のまま待機)。
No.13,14は遠心せず氷上に置いておく。


\subsection*{ヒスタミンの定量}

No.1~12については、ビペットマンのチップを液面近くに保ちながら(なるべく細胞をとらないように)上清を700$\mu$lずつ取り、
新しい1.5mlチューブに移す。
NO.13,14はしつかりサスペンド(ピペッティング)してから700$\mu$l取り、新しい1.5m1チューブに移す。
各チューブに1N HCl 50$\mu$lを加えて、ボルテックスし、さらに1N NaOH 250 $\mu$lを加えて、ボルテックスする。
o-Phthaldialdehyde(OPA)50$\mu$lを加えて、ボルテックスした後、
正確に4分間、室温に静置する(表6のタイムコースを参照)。3N HCl 100$\mu$lを加えて、ボルテックスし、15000rpm、4℃で5分間遠心する。
ピペットマンのチップを液面近くに保ちながら遠心後の上清を1mlずっ取り、新しい1.5ml工ッペンドルフチュープに移す。
200$\mu$lを96 well plateに移し、プレートリーダーで蛍光を測定する(励起波長:360nm、測定波長:450nm)
No.14をバックグラウンド、No.13の蛍光値をマスト細胞中の全てのヒスタミンによる蛍光と考え、
各サンプルではマスト細胞中のヒスタミンのうち何$\%$が放出されたか算出し、棒グラフを作成する。

\section*{結果}
血球計算盤を用いて顕微鏡で染色されたマスト細胞の数を数えたところ173個であった。
したがって、1mlあたりのマスト細胞の個数は$8.65 \times 10^5$cells/mlとなった。

ヒスタミンの定量の結果は以下のグラフのようになった。
\piccap{images/bars.png}{7}{ヒスタミンの定量}


\section*{考察}
LysoPSについては濃度の比例して蛍光度が観測されたことから正しく定量が行われたと考えられる。
一方LysoPEについては濃度と蛍光度に相関がない上に割合が大きすぎることから、マスト細胞が混入していたことが想定される。
また、標品のLysoPSの反応が悪かった。

\section*{課題}

\subsection*{生体膜脂質のオルガネラにおける偏在性について調べよ。}
\piccap{images/organella.png}{8}{生体脂質膜の組成}
\footnote{
出典 : http://book.bionumbers.org/what-lipids-are-most-abundant-in-membranes/
}
\begin{itemize}
\item 小胞体(ER)

PCが全体の半分を占め、ついでPE,PIが多い

\item 細胞膜(plasma membrane)

PCが一番多く、PE,SMがそれに次いでいる

\item ミトコンドリア(mitochondria)

PCとPEが多い

\item ゴルジ体(Golgi)

PCが最も多い
\end{itemize}

\subsection*{リゾリン脂質性メディエーターとその生理作用について調べよ。}
リン脂質のうち2本のアシル基から1本が取り除かれたものをリゾリン脂質といい、
近年新たな創薬ターゲットとして注目を集めている。
例えば、リゾホスファチジン酸は細胞膜上のLPA受容体に作用することで細胞増殖や抗アポトーシスといった作用をもたらす。
リゾホスファチジルセリンはGPR34受容体に作用しマスト細胞の脱顆粒や細胞の遊走を促進する。
\footnote{
出典 : http://lifesciencedb.jp/
}

\subsection*{OPAによってヒスタミンが定量できる原理を調べよ。}
図のようにOPA試薬がチオール化合物の存在下でヒスタミンと結合することで、イソインドール骨格を持った蛍光誘導体となり、定量が可能となる。
\footnote{
出典 : https://polaris.hoshi.ac.jp/kyoshitsu/bunseki/research2.html
}
\piccap{images/opa.png}{10}{OPA試薬による蛍光}

\subsection*{ConAによってマスト細胞が脱顆粒を起こす理由を調べよ。またIgEで感化されたマスト細胞が抗原により脱顆粒を起こす分子機構について調べよ。}
マスト細胞は、細胞表面のIgE受容体に結合したIgEが抗原により架橋されると脱顆粒し、ヒスタミンやセロトニンなどの生理活性物質を放出する。
しかしラット腹腔由来のマスト細胞では、抗原によるIgEの架橋だけでは脱顆粒を起こさず、リゾホスファチジルセリン(LysoPS)の添加を必要とする。この反応は他のリゾリン脂質では起こらず、LysoPSに特異性の高い反応である。

マウス腹腔内のホスファチジルセリン特異的ホスホリパーゼA1(PS-PLA1)がConA刺激によってラット腹腔細胞の細胞膜上に露出したPSを加水分解してLysoPSを産生してマスト細胞に供与する。供与されたLysoPSはその細胞表面に長時間結合し続け、その状態でIgE受容体が架橋されることが脱顆粒反応が起こるために必須であると参考文献には述べられている。炎症局所においては、PS-PLA1が存在すると同時に、アポトーシスやサイトカイン刺激により、PSが細胞表面に露出した細胞が多く見出される。このような場所で産生されたLysoPSがマスト細胞に作用すると、マスト細胞はIgE受容体架橋によって極めて脱顆粒しやすくなるとされる。
\footnote{
出典 : http://gakui.dl.itc.u-tokyo.ac.jp/cgi-bin/gazo.cgi?no=117433
}

\end{document}
